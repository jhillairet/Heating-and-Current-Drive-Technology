\chapter{Radio-Frequency Fundamentals}

\section{Introduction}
In this book we will the International System (SI) of units, which differs from most of the plasma physics and waves into plasma reference books in which CGS units are used instead. In this system of units, the unit of length is the meter, the unit of time is the second and the unit of mass is the kilogram. This choice is motivated by the fact that most engineering tools such as electromagnetic solvers also use SI units by default. Moreover, these units are also the one used in practice when performing measurements. 

\section{Basic Equations}
\subsection{Maxwell Equations}
We shall start the Maxwell equations is their most general form, before recasting them to fit our needs. The usual electromagnetic field quantities are expressed in terms of six quantities that are:
\begin{itemize}
 \item $\mathcal{E}$: the electric field intensity (in $V/m$)
 \item $\mathcal{H}$: the magnetic field intensity (in $A/m$)
 \item $\mathcal{D}$: the electric flux density (in $C/m^2$)
 \item $\mathcal{B}$: the magnetic flux density (in $Wb/m^2$)
 \item $\mathcal{J}$: the electric current density (in $A/m^2$)
 \item $\mathcal{Q}$: the electric charge density (in $C/m^3$)
\end{itemize}
where all quantities are function of space and time, e.g. $\mathcal{E}=\mathcal{E}(\mathbf{r},t)$.

Since James Clerk Maxwell discovered the full set of mathematical laws describing electromagnetic fields, many mathematicians, physicists and engineers have proposed different frameworks for representing fields and waves equations\parencite{Lindell2004, Warnick2014}.   
For day-to-day work in electromagnetic engineering, Heaviside's vector representation is commonly used. Within this frame,  Maxwell equations can be stated as a set of local differential equations:
\begin{subequations}
 \begin{align}
  \boldsymbol{\nabla} \times \boldsymbol{\mathcal{E}} &= -\frac{\partial \boldsymbol{\mathcal{B}}}{\partial t} \label{eq:Maxwell-Faraday}\\
  \boldsymbol{\nabla} \times \boldsymbol{\mathcal{H}} &= \frac{\partial \boldsymbol{\mathcal{D}}}{\partial t} + \boldsymbol{\mathcal{J}} \label{eq:Maxwell-Ampere} \\
  \boldsymbol{\nabla} \cdot \boldsymbol{\mathcal{D}} &= \mathcal{Q} \label{eq:Maxwell-Gauss} \\
  \boldsymbol{\nabla} \cdot \boldsymbol{\mathcal{B}} &= 0 \label{eq:Maxwell-Gauss-Magnetism}
 \end{align}
\end{subequations} 

The Maxwell-Faraday's law \ref{eq:Maxwell-Faraday} relates the magnetic flux to the electric field, by describing how a time-varying field induces an electric field.
The Maxwell-Ampere's law \ref{eq:Maxwell-Ampere} relates the current to the the magnetic field. It states that magnetic field can be generated by an electric current or by changing the electric field. The Maxwell-Gauss law \ref{eq:Maxwell-Gauss} describes the relationship between an electric flux density and the electric charges that cause it. The Maxwell-Gauss law for magnetism states that no magnetic charge exists as for electric charges.

In order to be able to apply the previous equations, we need to specify the relationships existing between electric, magnetic flux densities ($\mathcal{D}$,$\mathcal{B}$) and electric current density ($\mathcal{J}$) with electric and magnetic intensities ($\mathcal{E},\mathcal{H}$). These relations depend on the medium properties in which the field exists and are called \emph{constitutive relationships}. 

In vacuum or in any other medium having similar characteristics than vacuum (such as air), the constitutive relationships have their most simpler form:
\begin{subequations}
 \begin{align}
  \boldsymbol{\mathcal{D}} &= \varepsilon_0 \boldsymbol{\mathcal{E}} \\
  \boldsymbol{\mathcal{B}} &= \mu_0 \boldsymbol{\mathcal{H}} \\
  \boldsymbol{\mathcal{J}} &= 0
 \end{align}
\end{subequations}
where $\varepsilon_0$ is the vacuum \emph{permittivity} and $\mu_0$ the vacuum \emph{permeability}. 

In a standard isotropic linear mediums, the constitutive relationships becomes linear relationships:
\begin{subequations}
 \begin{align}
  \boldsymbol{\mathcal{D}} &= \varepsilon \boldsymbol{\mathcal{E}} \\
  \boldsymbol{\mathcal{B}} &= \mu \boldsymbol{\mathcal{H}} \\
  \boldsymbol{\mathcal{J}} &= \sigma \boldsymbol{\mathcal{E}}
 \end{align}
\end{subequations}
where $\varepsilon$ and $\mu$ are the medium permittivity and permeability respectively. The parameter $\sigma$ is called the \emph{conductivity} of the medium. Note that these relationships generally do not hold when the field intensities are very large or in time varying medium.

Simple matter medium can be classified according its values of $\varepsilon, \mu$ and $\sigma$. Materials with high conductivity value $\sigma$ are called \emph{conductors} while those having a small value are referred as \emph{dielectrics} or \emph{insulators}. In electromagnetic models, good conductors are often approximated to \emph{perfect conductors}, characterized by the limit $\sigma\to\infty$. On the other hand, \emph{perfect dielectrics} assume $\sigma=0$. 

The medium permittivity $\varepsilon$ can never be less than the vacuum permittivity $\varepsilon_0$. The \emph{relative permittivity} is defined such as $\varepsilon_r=\varepsilon/\varepsilon_0$. The permittivity of a conductor is hard to measure but appears to be unity\parencite{Harrington2001}. A similar definition holds for the \emph{relative permeability} $\mu_r=\mu/\mu_0$. For almost all materials except \emph{ferromagnetic} materials, one has $\mu=\mu_0$.

In anisotropic linear mediums, the constitutive relationships becomes tensor-relationships:
\begin{subequations}
 \begin{align}
  \boldsymbol{\mathcal{D}} &= \boldsymbol{\varepsilon} \cdot \boldsymbol{\mathcal{E}} \\
  \boldsymbol{\mathcal{B}} &= \boldsymbol{\mu} \cdot  \boldsymbol{\mathcal{H}} \\
  \boldsymbol{\mathcal{J}} &= \boldsymbol{\sigma} \cdot  \boldsymbol{\mathcal{E}}
 \end{align}
\end{subequations}
where $\boldsymbol{\varepsilon}$, $\boldsymbol{\mu}$ and $\boldsymbol{\sigma}$ are the dielectric tensor, the permeability tensor and the conductivity tensor respectively, which can be interpreted as 3x3 matrices\parencite{Swanson2003}.  

In general, the electromagnetic response of a medium is non-local with respect to both space and time\parencite{Mackay2010, Brambilla1998}. The medium response at the location $\mathbf{r}$ and time $t$ does not only depends on the field at location $\mathbf{r}$ and time $t$, but of the field in its vicinity $\mathbf{r}'$ and by all previous instant $t'$. Spacial non-locality can be significant when the wavelength is comparable to some characteristic length–scale in the medium. In plasma, the thermal agitation of the species induces add an additional erratic motion to the particles trajectory. Thus the particles are influenced by the field in the domain explored by their motion. In this situation, the constitutive relations of a linear medium should be stated as:
\begin{subequations}
 \begin{align}
  \boldsymbol{\mathcal{D}} = \int_{t'}\int_{\mathbf{r}'} &\left[
  \boldsymbol{\varepsilon}(\mathbf{r}', t') \cdot \boldsymbol{\mathcal{E}}(\mathbf{r},\mathbf{r}', t,t') + \right.\\
  & \left.   \boldsymbol{\nu}(\mathbf{r}', t') \cdot \boldsymbol{\mathcal{H}}(\mathbf{r},\mathbf{r}', t,t')  \right]d\mathbf{r}' dt' \nonumber \\
  \boldsymbol{\mathcal{B}}= \int_{t'}\int_{\mathbf{r}'} &\left[
    \boldsymbol{\xi}(\mathbf{r}', t') \cdot \boldsymbol{\mathcal{E}}(\mathbf{r},\mathbf{r}', t,t') + \right.\\
    & \left. \boldsymbol{\mu}(\mathbf{r}', t') \cdot \boldsymbol{\mathcal{H}}(\mathbf{r},\mathbf{r}', t,t') \right]  d\mathbf{r}' dt' \nonumber
 \end{align}
\end{subequations}



All of these relations hold only if the time rate of change of the electromagnetic field is small enough. Otherwise, one needs to extend the definition of linearity using linear differential relations\parencite{Harrington2001}:
\begin{subequations}
 \begin{align}
  \boldsymbol{\mathcal{D}} &= \varepsilon \boldsymbol{\mathcal{E}} + \varepsilon_1 \frac{\partial \boldsymbol{\mathcal{E}}}{\partial t} + \varepsilon_2 \frac{\partial^2 \boldsymbol{\mathcal{E}}}{\partial t^2} + \ldots \\
  \boldsymbol{\mathcal{B}} &= \varepsilon \boldsymbol{\mathcal{H}} + \varepsilon_1 \frac{\partial \boldsymbol{\mathcal{H}}}{\partial t} + \varepsilon_2 \frac{\partial^2 \boldsymbol{\mathcal{H}}}{\partial t^2} + \ldots \\
 \end{align}
\end{subequations}
Such situation arises typically when high intensity RF fields are used, which leads to non-linear phenomenons such \emph{ponderomotive effect}\parencite{Krapchev1979}.




\subsection{Time Harmonic Electromagnetic Fields}
\subsection{Phasors}
In most of the cases in magnetic fusion plasma heating and current drive, Radio-Frequency source time excitation varies sinusoidally in time with a single frequency (or \emph{AC} for Alternative Current). Such case of time varying electromagnetic fields is referred as \emph{time harmonic fields}. In this case, the mathematical analysis is simplified by using complex quantities. A scalar quantity $a$ can be defined as\footnote{The convention $ a \stackrel{\Delta}{=} \sqrt{2} |A| \sin (\omega t + \alpha) = \sqrt{2} \Im\left[A e^{j \omega t} \right]$ could also have been used.}:
\begin{equation}
 a \stackrel{\Delta}{=} \sqrt{2} |A| \cos (\omega t + \alpha) = \sqrt{2} \Re\left[A e^{j \omega t} \right] \label{eq:phasor}
\end{equation}
where $a=a(t,\mathbf{r})$ is called the \emph{instantaneous quantity} and $A=|A|e^{j\alpha}$ is called the \emph{complex quantity} or \emph{phasor}. Note that the complex quantity $A$ does \emph{not} depend of the time but it may be a function of position, i.e. $A=A(\mathbf{r})$. 

This definition, taken from \parencite{Harrington2001}, leads to few remarks. In electrical engineering, it is more practical to use time-averaged power (over any integer number of cycles) than instantaneous power since the voltages and currents are time-varying functions. The $\sqrt{2}$ factor, also known as the \emph{crest factor}, comes for the choice made for the magnitude $|A|$ of the complex quantity $A$ to be the \emph{effective} (or RMS, for Root-Mean-Square) value of the sinusoidally time varying quantity $\mathcal{A}$\footnote{Can also be proved from the average of the product of two instantaneous complex quantities $a$ and $b$ as defined by (\ref{eq:phasor}) which is  
$$ 
\frac{1}{T}  \int_0^T a(t) b(t)\diff t = \Re[AB^*]
$$. Then, one deduces that 
$$
\frac{1}{T}  \int_0^T \left[a(t)\right]^2 \diff t = \Re[AA^*] = |A|^2
$$}: 
\begin{subequations}
 \begin{align}
  \mathcal{A}_{\mathrm{rms}}
    &= \sqrt{\frac{1}{T} \int_0^T \left[\mathcal{A}(t) \right]^2 \diff t}  \nonumber \\
    &= \sqrt{\frac{1}{T} \int_0^T  \left[\sqrt{2} |A| \cos (\omega t + \alpha) \right]^2 \diff t} \nonumber \\
    &= \sqrt{\frac{|A|^2}{T} \int_0^T  \left[ 1+\cos (2\omega t + 2\alpha) \right] \diff t} \nonumber \\
    &= |A| \nonumber
 \end{align}
\end{subequations}
Dropping the factor $\sqrt{2}$ in (\ref{eq:phasor}) would lead to express $|A|$ as the peak value of $\mathcal{A}$ instead\footnote{Which is the case for example in ANSYS HFSS (while the Poynting vector is a time-averaged quantity)}. 

When calculating the complex power, one advantage of the previous definition is to get the same proportionally factors as for their instantaneous counterparts, i.e. $p=v i$ for the instantaneous power and $P=VI^*$ for the complex power. Otherwise, a factor $1/2$ would appear in the complex power if peak values would have been used for $|V|$ and $|I|$.

This definition can be extended to vectors quantities having sinusoidal time variation:
\begin{equation}
 \boldsymbol{\mathcal{E}} = \sqrt{2} \Re \left[ \mathbf{E} e^{j\omega t}\right]
\end{equation}
which means that each components of $\mathbf{E}$ are related to the components of $ \boldsymbol{\mathcal{E}}$ by the relation (\ref{eq:phasor}). For example, in a cartesian frame, the components of $\boldsymbol{\mathcal{E}}$ and $\mathbf{E}$ are related by:
\begin{eqnarray*}
 \mathcal{E}_x = \sqrt{2} \Re \left[ E_x e^{j\omega t}\right] = \sqrt{2} |E_x| \cos (\omega t + \phi_x) \\
 \mathcal{E}_y = \sqrt{2} \Re \left[ E_y e^{j\omega t}\right] = \sqrt{2} |E_y| \cos (\omega t + \phi_y) \\
 \mathcal{E}_z = \sqrt{2} \Re \left[ E_z e^{j\omega t}\right] = \sqrt{2} |E_z| \cos (\omega t + \phi_z) 
\end{eqnarray*}
which leads to:
\begin{subequations}
 \begin{align}
  \mathcal{E}_{\mathrm{rms}} 
    =& \sqrt{\frac{1}{T} \int_0^T  \left[ \boldsymbol{\mathcal{E}}(t) \right]^2 \diff t} \nonumber \\
    =& \sqrt{\frac{1}{T} \int_0^T  \left[ \boldsymbol{\mathcal{E}}(t) \cdot \boldsymbol{\mathcal{E}}(t) \right] \diff t} \nonumber \\
    =& \sqrt{\frac{1}{T} \int_0^T  \left[ \mathcal{E}_x^2 + \mathcal{E}_y^2 + \mathcal{E}_z^2  \right] \diff t} \nonumber \\
    =& \sqrt{|E_x|^2 + |E_y|^2 + |E_z|^2} \nonumber \\
    =& \sqrt{ \mathbf{E} \cdot \mathbf{E}^* } \nonumber \\
    =& \left| \mathbf{E} \right| \nonumber
 \end{align}
\end{subequations}

Note that the phases $\phi_x, \phi_y, \phi_z$ are not necessarily equal. This leads to an important remark on the evaluation of peak values of time-harmonic vector fields. For sinusoidally time varying \emph{scalar} complex quantity, the \emph{peak} value can be obtained from:
\begin{equation}
 A_{\mathrm{peak}} = \sqrt{2} |A|
\end{equation}
However, this relation does not hold in general for vector fields, unless in the particular case of linearly polarized fields\footnote{Inversely, if we have defined (\ref{eq:phasor}) without the $\sqrt{2}$ factor, the rms value would not be in general derived from $1/\sqrt{2}$ the peak value.}. In the case of circularly polarized field for example, the peak value is constant in time and such is equal to the rms value\parencite{Faria2008}.


Finally, note that in electrical engineering, the time convention in (\ref{eq:phasor}) is usually $e^{j\omega t}$ while in physics $e^{-j\omega t}$ is preferred\parencite{Bradley2007, Michelsen2017}. This choice is motivated by the fact that most of the electromagnetic solver packages use the former convention, and so to avoid any confusion. Note however that most of the physics books on plasma waves adopt instead the later convention \parencite{Swanson2003, Stix1992, Brambilla1998}.


\subsection{Time Harmonic Maxwell Equations}
For time-harmonic fields, using phasor analysis leads to obtain single frequency steady state response. Using the mathematical properties of the real part operator $\Re$, Maxwell equations can be reformulated in terms of complex phasors. See sec.1-8 of \parencite{Harrington2001} for a complete derivation. Moreover, in this process, time-derivatives are expressed by a $j\omega$ multiplier.  

Thus, the Maxwell equations (\ref{eq:Maxwell-Faraday}, \ref{eq:Maxwell-Ampere}, \ref{eq:Maxwell-Gauss} \ref{eq:Maxwell-Gauss-Magnetism}) become on their complex form:
\begin{subequations}
 \begin{align}
  \boldsymbol{\nabla} \times \mathbf{H} &= j\omega\mathbf{D} + \mathbf{J}  \label{eq:Maxwell-Faraday-Harmonic} \\
  \boldsymbol{\nabla} \times \mathbf{E} &= -j\omega\mathbf{B} \label{eq:Maxwell-Ampere-Harmonic} \\
  \boldsymbol{\nabla} \cdot \mathbf{D} &= Q \label{eq:Maxwell-Gauss-Harmonic} \\
  \boldsymbol{\nabla} \cdot \mathbf{B} &= 0 \label{eq:Maxwell-Gauss-Magnetism-Harmonic} 
 \end{align}
\end{subequations}

The constitutive relationships become in AC:
\begin{subequations}
 \begin{align}
  \mathbf{D} =& \boldsymbol{\varepsilon}(\omega) \mathbf{E} \\
  \mathbf{B} =& \boldsymbol{\mu}(\omega) \mathbf{H} \\
  \mathbf{J} =& \boldsymbol{\sigma}(\omega) \mathbf{E}
 \end{align}
\end{subequations}
where $\boldsymbol{\varepsilon}(\omega)$ is the \emph{complex permittivity}, $\boldsymbol{\mu}(\omega)$ the \emph{complex permeability} and $\boldsymbol{\sigma}(\omega)$ the \emph{complex conductivity} of the media. 

Note the previous equations do not depend of time but of frequency $\omega$. Finally, when the time-excitation has an arbitrary time dependence, electromagnetic fields can be determined by superposition principle (Fourier transform). 

Expressing the previous equations in terms of $\mathbf{E}$ and $\mathbf{H}$ only leads to:
\begin{subequations}
 \begin{align}
  \boldsymbol{\nabla} \times \mathbf{H} &= \left(j\omega\boldsymbol{\varepsilon}(\omega) + \boldsymbol{\sigma}(\omega) \right) \mathbf{E}  \label{eq:Maxwell-Faraday-Harmonic} \\
  \boldsymbol{\nabla} \times \mathbf{E} &= -j\omega\boldsymbol{\mu}(\omega) \mathbf{H}  \label{eq:Maxwell-Ampere-Harmonic} \\
  \boldsymbol{\nabla} \cdot (\boldsymbol{\varepsilon}(\omega) \mathbf{E}) &= Q \label{eq:Maxwell-Gauss-Harmonic} \\
  \boldsymbol{\nabla} \cdot (\boldsymbol{\mu}(\omega) \mathbf{H} ) &= 0 \label{eq:Maxwell-Gauss-Magnetism-Harmonic} 
 \end{align}

\end{subequations}


\section{Transmission Line Theory}

\section{Microwave safety}
\parencite[sec.5.8.3]{Benford2015}


