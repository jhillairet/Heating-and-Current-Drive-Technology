\documentclass[a4paper,10pt]{book}
\usepackage[utf8]{inputenc}
\usepackage[english]{babel}
\usepackage{gensymb} % for ^{\circ} or \degree for degrees C
\usepackage{graphicx}
\usepackage[table,x11names]{xcolor}
\usepackage{tabularx} % smarter arrays
\usepackage{amsmath}
\usepackage{esint} % \oiint

% Bibliography configuration
\usepackage[
    backend=biber, 
    style=authoryear,
    url=false, 
    doi=true
]{biblatex}
\addbibresource{HCD_Textbook.bib}

\usepackage[]{hyperref}
\hypersetup{
    colorlinks=true,
}
% Add brackets [] around citations when using \parencite{} and authoryear format
\newrobustcmd*{\parentexttrack}[1]{%
  \begingroup
  \blx@blxinit
  \blx@setsfcodes
  \blx@bibopenparen#1\blx@bibcloseparen
  \endgroup}

\AtEveryCite{%
  \let\parentext=\parentexttrack%
  \let\bibopenparen=\bibopenbracket%
  \let\bibcloseparen=\bibclosebracket}

\title{Heating and Current Drive Science and Technology for Controlled Magnetic Fusion}

% convenient differential operator typesetting according to 
% https://tex.stackexchange.com/questions/60545/should-i-mathrm-the-d-in-my-integrals
\newcommand*\diff{\mathop{}\!\mathrm{d}}
\newcommand*\Diff[1]{\mathop{}\!\mathrm{d^#1}}
\newcommand{\E}{\mathbf{E}}
\newcommand{\calE}{\boldsymbol{\mathcal{E}}}
\renewcommand{\H}{\mathbf{H}}
\newcommand{\calH}{\boldsymbol{\mathcal{H}}}
\newcommand{\D}{\mathbf{D}}
\newcommand{\calD}{\boldsymbol{\mathcal{D}}}
\newcommand{\B}{\mathbf{B}}
\newcommand{\calB}{\boldsymbol{\mathcal{B}}}
\renewcommand{\k}{\mathbf{k}}
\renewcommand{\r}{\mathbf{r}}
\newcommand{\eps}{\boldsymbol{\varepsilon}}
\newcommand{\mut}{\boldsymbol{\mu}}
\newcommand{\del}{\boldsymbol{\nabla}}

% List of Authors
% List of authors. 
% 
% If you add your name to this listm please use the following 
%
%\author{
%  LastName1, FirstName1\\
%  \texttt{first1.last1@xxxxx.com}
%  \and
%  LastName2, FirstName2\\
%  \texttt{first2.last2@xxxxx.com}
%}

\author{Hillairet, Julien\\
  \texttt{julien.hillairet@cea.fr}
  }

\begin{document}
\maketitle
\tableofcontents

%\chapter{Introduction}
\chapter{Radio-Frequency Fundamentals}
% ###########################################################################
% ###########################################################################
% ###########################################################################
\section{Introduction}
In this book we will the International System (SI) of units, which differs from most of the plasma physics and waves into plasma reference books in which CGS units are used instead. In this system of units, the unit of length is the meter, the unit of time is the second and the unit of mass is the kilogram. This choice is motivated by the fact that most engineering tools such as electromagnetic solvers also use SI units by default. Moreover, these units are also the one used in practice when performing measurements. 

% ###########################################################################
% ###########################################################################
% ###########################################################################
\section{Basic Equations}
% ###########################################################################
% ###########################################################################
\subsection{Maxwell Equations}
We shall start the Maxwell equations is their most general form, before recasting them to fit our needs. The usual electromagnetic field quantities are expressed in terms of six quantities that are:
\begin{itemize}
 \item $\boldsymbol{\mathcal{E}}$: the electric field intensity (in $V/m$)
 \item $\boldsymbol{\mathcal{H}}$: the magnetic field intensity (in $A/m$)
 \item $\boldsymbol{\mathcal{D}}$: the electric flux density (in $A\cdot s/m^2=C/m^2$)
 \item $\boldsymbol{\mathcal{B}}$: the magnetic flux density (in $V\cdot s/m^2=Wb/m^2$, also known as Tesla $T$)
 \item $\boldsymbol{\mathcal{J}}$: the electric current density (in $A/m^2$)
 \item $q_v$: the electric charge density (in $C/m^3$)
\end{itemize}
where all quantities are function of space and time, e.g. $\boldsymbol{\mathcal{E}}=\boldsymbol{\mathcal{E}}(\mathbf{r},t)$.

Since James Clerk Maxwell discovered the full set of mathematical laws describing electromagnetic fields, many mathematicians, physicists and engineers have proposed different frameworks for representing fields and waves equations\parencite{Lindell2004, Warnick2014}. For day-to-day work in electromagnetic engineering, Heaviside's vector representation is commonly used. Within this frame,  Maxwell equations can be stated as a set of local differential equations:
\begin{subequations}
 \begin{align}
  \boldsymbol{\nabla} \times \boldsymbol{\mathcal{E}} &= -\frac{\partial \boldsymbol{\mathcal{B}}}{\partial t} \label{eq:Maxwell-Faraday}\\
  \boldsymbol{\nabla} \times \boldsymbol{\mathcal{H}} &= \frac{\partial \boldsymbol{\mathcal{D}}}{\partial t} + \boldsymbol{\mathcal{J}} \label{eq:Maxwell-Ampere} \\
  \boldsymbol{\nabla} \cdot \boldsymbol{\mathcal{D}} &= q_v \label{eq:Maxwell-Gauss} \\
  \boldsymbol{\nabla} \cdot \boldsymbol{\mathcal{B}} &= 0 \label{eq:Maxwell-Gauss-Magnetism}
 \end{align}
 \label{eq:MaxwellEquations}
\end{subequations} 

The Maxwell-Faraday's law \ref{eq:Maxwell-Faraday} relates the magnetic flux to the electric field, by describing how a changing magnetic flux induces an electric field.
The Maxwell-Ampere's law \ref{eq:Maxwell-Ampere} relates the current to the magnetic field. It states that magnetic field can be generated by a changing electric flux density and by electric current. 
The Maxwell-Gauss law \ref{eq:Maxwell-Gauss} describes the relationship between an electric flux density and the electric charges that cause it. 
The Maxwell-Gauss law for magnetism states that no magnetic charge exists as for electric charges.

The corresponding equations in integral form are:
\begin{subequations}
	\begin{align}
	\oint \boldsymbol{\mathcal{E}} \cdot \diff \mathbf{l} 
	=&
		- \frac{\diff }{\diff t} \iint \boldsymbol{\mathcal{B}} \cdot \diff \mathbf{S} 
		\\
		\oint \boldsymbol{\mathcal{B}} \cdot \diff \mathbf{l} 
		=&
		\frac{\diff }{\diff t} \iint \boldsymbol{\mathcal{D}} \cdot \diff \mathbf{S} 	
		+ \iint \boldsymbol{\mathcal{J}} \cdot \diff \mathbf{S} 
		\\
		\oiint \boldsymbol{\mathcal{B}} \cdot \diff \mathbf{S} 
		=& \; 0
		\\
		\oiint \boldsymbol{\mathcal{D}} \cdot \diff \mathbf{S} 
		=& 
		\iiint q_v \diff v			
	\end{align}
	\label{eq:MaxwellEquationsIntegral}
\end{subequations}

On can define the following \emph{circuit} quantities associated with each field quantities\parencite{Harrington2001}:
\begin{itemize}
	\item $v$, the \emph{voltage} in $V$
	\item $i$, the \emph{current} in $A$
	\item $q$, the \emph{electric charge} in $C$
	\item $\psi$, the \emph{magnetic flux} in $Wb$
	\item $\psi_e$, the \emph{electric flux} in $C$
	\item $u$, the \emph{magnetomotive force} in $A$
\end{itemize}
defined by:
\begin{subequations}
	\begin{align}
	v =& \int \boldsymbol{\mathcal{E}} \cdot \diff \mathbf{l} \\
	i =& \iint \boldsymbol{\mathcal{J}} \cdot \diff \mathbf{S} \\
	q =& \iiint q_v \diff v \\
	\psi =& \iint \boldsymbol{\mathcal{B}} \cdot \diff \mathbf{S} \\
	\psi_e=& \iint \boldsymbol{\mathcal{D}} \cdot \diff \mathbf{S} \\
	u =& \int \boldsymbol{\mathcal{H}} \cdot \diff \mathbf{l}
	\end{align}
	\label{eq:CircuitQuantities}
\end{subequations}


In order to be able to solve the  equations(\ref{eq:MaxwellEquations}), one needs to specify the relationships existing between electric, magnetic flux densities ($\mathcal{D}$,$\mathcal{B}$) and electric current density $\mathcal{J}$ with electric and magnetic intensities ($\mathcal{E},\mathcal{H}$)\footnote{
	I've chosen here is to treat both $\mathbf{E}$ and $\mathbf{H}$ as similar kinds of fields, different from $\mathbf{D}$ and $\mathbf{B}$, as for example it is done in \parencite{Pozar1998} or \parencite{Harrington2001}. This choice has symbolic advantages when dealing with RF networks with SI units, where the electric and magnetic field intensity can be associated to voltages and currents as they are measured (tbc).
	
	However, many (most?) authors have chosen instead to treat $\mathbf{E}$ and $\mathbf{B}$ as the \emph{fundamental} quantities and $\mathbf{D}$ and $\mathbf{H}$ as \textit{auxiliary} (or convenience) quantities denoting the average fields over macroscopic small regions\parencite{Lindell1995, Griffiths2005, Jackson1999}. Indeed, both $\mathbf{D}$ and $\mathbf{H}$ allow to write Gauss and Ampere laws in terms of the \emph{free} charges and currents alone, such "incorporating" \emph{bounds} charge and current contributions. Thus, in these macroscopic Maxwell equations, only the external charges and currents brought into the system from outside are considered, without taking care of the average charge and current distributions in the medium, which can be a convenient mathematical tool. This choice has sense, this one cannot turn off bounds contributions as he can for free contributions \parencite[sec.6.3]{Griffiths2005}. In such as case, one would have read instead $\mathbf{H}=\boldsymbol{\mu}^{-1} $. Finally, the magnetic flux density $\mathbf{B}$ can be derived as the character of an electric field in a co-moving frame\parencite{Schwinger1998}.
	
	The latter choice can also take origin from the interpretation that the electric and magnetic fields as a force acting on a test charge via the Lorentz force $q(\mathbf{E}+\mathbf{v}\times\mathbf{B})$, leading naturally to $(\mathbf{E}$,$\mathbf{B})$. However, if one uses an energy picture using differential forms, interpreting the electromagnetic field as the change of energy experienced by a test charge as it moves through the field, the couples $(\mathbf{E}$,$\mathbf{H})$ and $(\mathbf{D}$,$\mathbf{B})$ are represented by different mathematical objects\parencite{Warnick2014}. 
	}
. These relations depend on the medium properties in which the field exists and are called \emph{constitutive relations}. Explicit forms of these relationships have been found from experimentation or deduced from atomic considerations \parencite[sec.5]{Schwinger1998} and are discussed in a later section.


% ###########################################################################
% ###########################################################################
% ###########################################################################
\subsection{Time Harmonic Electromagnetic Fields}
% ###########################################################################
% ###########################################################################
\subsubsection{Phasors}
In most of the cases in magnetic fusion plasma heating and current drive, Radio-Frequency source time excitation varies sinusoidally in time with a single frequency (or \emph{AC} for Alternative Current). Such case of time varying electromagnetic fields is referred as \emph{time harmonic} or \emph{monochromatic} fields. In this case, the mathematical analysis is simplified by using complex quantities. A scalar quantity $a$ can be defined as\footnote{The convention $ a \stackrel{\Delta}{=} \sqrt{2} |A| \sin (\omega t + \alpha) = \sqrt{2} \Im\left[A e^{j \omega t} \right]$ could also have been used.}:
\begin{equation}
a(\mathbf{r},t)
\stackrel{\Delta}{=} 
\sqrt{2} |A(\mathbf{r})| \cos (\omega t + \alpha) = \sqrt{2} \Re\left[A(\mathbf{r}) e^{j \omega t} \right] \label{eq:phasor}
\end{equation}
where $a=a(\mathbf{r}, t)$ is called the \emph{instantaneous quantity} and $A=|A|e^{j\alpha}$ is called the \emph{complex quantity} or \emph{phasor}. Note that the complex quantity $A$ does \emph{not} depend of the time but it may be a function of position, i.e. $A=A(\mathbf{r})$. 

This definition, taken from \parencite{Harrington2001}, leads to few remarks. In electrical engineering, it is more practical to use time-averaged power (over any integer number of cycles) than instantaneous power since the voltages and currents are time-varying functions. The $\sqrt{2}$ factor, also known as the \emph{crest factor}, comes for the choice made for the magnitude $|A|$ of the complex quantity $A$ to be the \emph{effective} (or RMS, for Root-Mean-Square) value of the sinusoidally time varying quantity $\mathcal{A}$\footnote{Can also be proved from the average of the product of two instantaneous complex quantities $a$ and $b$ as defined by (\ref{eq:phasor}) which is  
	$$ 
	\frac{1}{T}  \int_0^T a(t) b(t)\diff t = \Re[AB^*]
	$$. Then, one deduces that 
	$$
	\frac{1}{T}  \int_0^T \left[a(t)\right]^2 \diff t = \Re[AA^*] = |A|^2
	$$}: 
\begin{subequations}
	\begin{align}
	a_{\mathrm{rms}}
	&= \sqrt{\frac{1}{T} \int_0^T \left[a(t) \right]^2 \diff t}  \nonumber \\
	&= \sqrt{\frac{1}{T} \int_0^T  \left[\sqrt{2} |A| \cos (\omega t + \alpha) \right]^2 \diff t} \nonumber \\
	&= \sqrt{\frac{|A|^2}{T} \int_0^T  \left[ 1+\cos (2\omega t + 2\alpha) \right] \diff t} \nonumber \\
	&= |A| \nonumber
	\end{align}
\end{subequations}
Dropping the factor $\sqrt{2}$ in (\ref{eq:phasor}) would lead to express $|A|$ as the peak value of $\mathcal{A}$ instead\footnote{Which is the case for example in ANSYS HFSS (while the Poynting vector is a time-averaged quantity)}. 

When calculating the complex power, one advantage of the previous definition is to get the same proportionally factors as for their instantaneous counterparts, i.e. $p=v i$ for the instantaneous power and $P=VI^*$ for the complex power. Otherwise, a factor $1/2$ would appear in the complex power if peak values would have been used for $|V|$ and $|I|$.

This definition can be extended to vectors quantities having sinusoidal time variation:
\begin{equation}
\boldsymbol{\mathcal{E}}(\mathbf{r},t) = \sqrt{2} \Re \left[ \mathbf{E}(\mathbf{r}) e^{j\omega t}\right]
\label{eq:definitionTimeHarmonicField}
\end{equation}
which means that each components of $\mathbf{E}$ are related to the components of $ \boldsymbol{\mathcal{E}}$ by the relation (\ref{eq:phasor}). For example, in a cartesian frame, the components of $\boldsymbol{\mathcal{E}}$ and $\mathbf{E}$ are related by:
\begin{eqnarray*}
	\mathcal{E}_x = \sqrt{2} \Re \left[ E_x e^{j\omega t}\right] = \sqrt{2} |E_x| \cos (\omega t + \phi_x) \\
	\mathcal{E}_y = \sqrt{2} \Re \left[ E_y e^{j\omega t}\right] = \sqrt{2} |E_y| \cos (\omega t + \phi_y) \\
	\mathcal{E}_z = \sqrt{2} \Re \left[ E_z e^{j\omega t}\right] = \sqrt{2} |E_z| \cos (\omega t + \phi_z) 
\end{eqnarray*}
which leads to:
\begin{subequations}
	\begin{align}
	\mathcal{E}_{\mathrm{rms}} 
	=& \sqrt{\frac{1}{T} \int_0^T  \left[ \boldsymbol{\mathcal{E}}(t) \right]^2 \diff t} \nonumber \\
	=& \sqrt{\frac{1}{T} \int_0^T  \left[ \boldsymbol{\mathcal{E}}(t) \cdot \boldsymbol{\mathcal{E}}(t) \right] \diff t} \nonumber \\
	=& \sqrt{\frac{1}{T} \int_0^T  \left[ \mathcal{E}_x^2 + \mathcal{E}_y^2 + \mathcal{E}_z^2  \right] \diff t} \nonumber \\
	=& \sqrt{|E_x|^2 + |E_y|^2 + |E_z|^2} \nonumber \\
	=& \sqrt{ \mathbf{E} \cdot \mathbf{E}^* } \nonumber \\
	=& \left| \mathbf{E} \right| \nonumber
	\end{align}
\end{subequations}

Note that the phases $\phi_x, \phi_y, \phi_z$ are not necessarily equal. This leads to an important remark on the evaluation of peak values of time-harmonic vector fields. For sinusoidally time varying \emph{scalar} complex quantity, the \emph{peak} value can be obtained from:
\begin{equation}
A_{\mathrm{peak}} = \sqrt{2} |A|
\end{equation}
However, this relation does not hold in general for vector fields, unless in the particular case of linearly polarized fields\footnote{Inversely, if we have defined (\ref{eq:phasor}) without the $\sqrt{2}$ factor, the rms value would not be in general derived from $1/\sqrt{2}$ the peak value.}. In the case of circularly polarized field for example, the peak value is constant in time and such is equal to the rms value\parencite{Faria2008}.


Finally, note that in electrical engineering, the time convention in (\ref{eq:phasor}) is usually $e^{j\omega t}$ while in physics $e^{-j\omega t}$ is preferred\parencite{Bradley2007, Michelsen2017}. This choice is motivated by the fact that most of the electromagnetic solver packages use the former convention, and so to avoid any confusion. Note however that most of the physics books on plasma waves adopt instead the later convention \parencite{Swanson2003, Stix1992, Brambilla1998}.




% ###########################################################################
% ###########################################################################
\subsubsection{Time Harmonic Maxwell Equations}
For time-harmonic fields, using phasor analysis leads to obtain single frequency steady state response. Using the mathematical properties of the real part operator $\Re$, Maxwell equations can be reformulated in terms of complex phasors (See sec.1-8 of \parencite{Harrington2001} for a complete derivation). Moreover, in this process, time-derivatives are expressed by a $j\omega$ multiplier.  

Thus, the Maxwell equations (\ref{eq:Maxwell-Faraday}, \ref{eq:Maxwell-Ampere}, \ref{eq:Maxwell-Gauss} \ref{eq:Maxwell-Gauss-Magnetism}) become for time harmonic fields:
\begin{subequations}
	\begin{align}
	\boldsymbol{\nabla} \times \mathbf{H} (\mathbf{r}) &= j\omega\mathbf{D}(\mathbf{r}) + \mathbf{J}(\mathbf{r})  \label{eq:Maxwell-Faraday-Harmonic} \\
	\boldsymbol{\nabla} \times \mathbf{E} (\mathbf{r}) &= -j\omega\mathbf{B}(\mathbf{r}) \label{eq:Maxwell-Ampere-Harmonic} \\
	\boldsymbol{\nabla} \cdot \mathbf{D} (\mathbf{r}) &= q_v(\mathbf{r}) \label{eq:Maxwell-Gauss-Harmonic} \\
	\boldsymbol{\nabla} \cdot \mathbf{B} (\mathbf{r}) &= 0 \label{eq:Maxwell-Gauss-Magnetism-Harmonic} 
	\end{align}
	\label{eq:MaxwellEquationsTimeHarmonic}
\end{subequations}

% #######################################
\subsubsection{Finite bandwidth solutions}
Let us generalize to field solutions of finite bandwidth, which are a continuous distribution of frequencies $\omega$. Similarly to our definition  (\ref{eq:definitionTimeHarmonicField}), we express the field by a summation of time-harmonic solutions over all frequencies:
\begin{equation}
\boldsymbol{\mathcal{E}}(\mathbf{r}, t)
=
\int
\Re \left[
\boldsymbol{\mathcal{E}}(\mathbf{r}, \omega)
e^{j\omega t}
\right] 
\diff \omega
\end{equation}
The real part operator can be put outside the integral:
\begin{equation}
\boldsymbol{\mathcal{E}}(\mathbf{r}, t)
=
\Re \left[
\int
\boldsymbol{\mathcal{E}}(\mathbf{r}, \omega)
e^{j\omega t}
\diff \omega
\right] 
\end{equation}
which then can be seen as a time-domain Fourier transforms, defined such as as:
\begin{subequations}
	\begin{align}
		f(t) =& \int_{-\infty}^{+\infty} f(\omega) e^{j\omega t} \diff \omega \\
	f(\omega) =& \frac{1}{2\pi}\int_{-\infty}^{+\infty} f(t) e^{-j\omega t} \diff t 
	\end{align}
	\label{eq:FourierTransformTime}
\end{subequations}
If $f(t)$ is a real function, then a property of the Fourier transform for $f(\omega)$ is\footnote{More of Fourier transform in \parencite[sec.1.3]{Clemmow1996} and \parencite[chap.4]{Harrington2001}.}:
\begin{equation}
f(-\omega) = f^*(\omega)
\label{eq:FourierPropertyRealFunction}
\end{equation}


If we require the property  (\ref{eq:FourierPropertyRealFunction}) to $\boldsymbol{\mathcal{E}}(\omega)$, i.e. $
\boldsymbol{\mathcal{E}}(\mathbf{r}, -\omega)
\overset{\Delta}{=}
\boldsymbol{\mathcal{E}}^*(\mathbf{r}, -\omega)
$
then we can eliminate the real part operator $\Re$:
\begin{equation}
\boldsymbol{\mathcal{E}}(\mathbf{r}, t)
=
\int
\boldsymbol{\mathcal{E}}(\mathbf{r}, \omega)
e^{j\omega t}
\diff \omega
\end{equation}
so $\boldsymbol{\mathcal{E}}(\mathbf{r}, t)$ forms a Fourier transform pair with $\boldsymbol{\mathcal{E}}(\mathbf{r}, \omega)$,  defined by:
\begin{equation}
	\boldsymbol{\mathcal{E}}(\mathbf{r}, \omega) = \frac{1}{2\pi}\int_{-\infty}^{+\infty} 
	\boldsymbol{\mathcal{E}}(\mathbf{r}, t)
	 e^{-j\omega t} \diff t 
	 \label{eq:TimeFourierTransformE}
\end{equation}

After introducing the Fourier transform to Maxwell equations (\ref{eq:MaxwellEquations}), one obtains the time-spectral representation of Maxwell equations:
\begin{subequations}
	\begin{align}
	\boldsymbol{\nabla} \times \boldsymbol{\mathcal{H}} (\mathbf{r}, \omega) &= j\omega\boldsymbol{\mathcal{D}} (\mathbf{r}, \omega) + \boldsymbol{\mathcal{J}} (\mathbf{r}, \omega)   \\
	\boldsymbol{\nabla} \times \boldsymbol{\mathcal{E}} (\mathbf{r}, \omega) &= -j\omega\boldsymbol{\mathcal{B}} (\mathbf{r}, \omega)  \\
	\boldsymbol{\nabla} \cdot \boldsymbol{\mathcal{D}} (\mathbf{r}, \omega) &= q_v(\mathbf{r}, \omega)  \\
	\boldsymbol{\nabla} \cdot \boldsymbol{\mathcal{B}} (\mathbf{r}, \omega) &= 0  
	\end{align}
	\label{eq:MaxwellEquationTimeSpectral}
\end{subequations}
which have the same form than (\ref{eq:MaxwellEquationsTimeHarmonic}), but with different quantities since they do not have different units\parencite[sec.1.7.5]{Smith1997}. For example, the electric field (phasor) in (\ref{eq:Maxwell-Ampere-Harmonic}) was in $V/m$, while here it is in $V/m/Hz$ that is intensity per unit frequency. 

We can recover (\ref{eq:MaxwellEquationsTimeHarmonic}) in the case of time-harmonic field, i.e if $\boldsymbol{\mathcal{E}}(\mathbf{r}, t) = \mathbf{E}(\mathbf{r})\cos\omega_0 t=\Re[\mathbf{E}(\mathbf{r})e^{j\omega_0 t}]$, where (\ref{eq:TimeFourierTransformE}) becomes\footnote{We have dropped the $\sqrt{2}$ for convenience but without loss of generality.}:
\begin{eqnarray}
\boldsymbol{\mathcal{E}}(\mathbf{r},\omega) 
	&=&  
	\frac{1}{2\pi}\int_{-\infty}^{+\infty} \diff t \;
	\Re[\mathbf{E}(\mathbf{r})e^{j\omega_0 t}]
	e^{-j\omega t} 
	\nonumber
	\\
	&=&  
	\frac{1}{4\pi}\int_{-\infty}^{+\infty} \diff t \;
	\mathbf{E}(\mathbf{r}) e^{-j(\omega-\omega_0 ) t}
	+
	\mathbf{E}^*(\mathbf{r}) e^{-(\omega+\omega_0 ) t} 
	\nonumber
	\\
	&=&
	\frac{1}{4\pi} \left[
	\mathbf{E}(\mathbf{r}) \delta(\omega - \omega_0) 
	+
	\mathbf{E}^*(\mathbf{r}) \delta(\omega + \omega_0)  
	\right]
	\nonumber
\end{eqnarray}
where we have the Dirac distribution definition:
\begin{eqnarray}
\delta(x - x_0) 
= \frac{1}{2\pi}
\int
e^{j (x-x_0) u} \diff u
\label{eq:DiracDefinition}
\end{eqnarray}
Moreover, using the property (\ref{eq:FourierPropertyRealFunction}) and the fact that the Dirac distribution is even, we have finally:
\begin{eqnarray}
\boldsymbol{\mathcal{E}}(\mathbf{r},\omega) 
&=&
\frac{1}{4\pi}\int_{-\infty}^{+\infty} \diff t \;
\mathbf{E}(\mathbf{r}) e^{-j(\omega-\omega_0 ) t}
+
\frac{1}{4\pi}\int_{-\infty}^{+\infty} \diff t
\mathbf{E}(-\mathbf{r}) e^{-j(\omega+\omega_0 ) t} 
\nonumber
\\
&=&
\frac{1}{4\pi}\int_{-\infty}^{+\infty} \diff t \;
\mathbf{E}(\mathbf{r}) e^{-j(\omega-\omega_0 ) t}
+
\frac{1}{4\pi}\int_{-\infty}^{+\infty} \diff t
\mathbf{E}(\mathbf{r}) e^{+j(\omega+\omega_0 ) t} 
\nonumber
\\
&=&
\frac{1}{2\pi}
\mathbf{E}(\mathbf{r}) \delta(\omega - \omega_0) 
\end{eqnarray}
Doing the same for other fields and plugging into (\ref{eq:MaxwellEquationTimeSpectral}) leads to recover (\ref{eq:MaxwellEquationsTimeHarmonic}) if sources and materials do not depend of the frequency (no time dispersion). 


% #######################################
\subsection{$k-\omega$ Fields Representation}\label{sec:spectralRepresentation}
Let us generalise to the case where the field solution is represented by the summation of many plane (or evanescent) waves characterized by their wavevector $\mathbf{k}=(k_x, k_y, k_z)$. We construct a solution on the form: 	
\begin{subequations}
	\begin{align}
	\boldsymbol{\mathcal{A}}(\mathbf{r}, t) =& \Re \left[
	\int \diff \omega \int \diff \mathbf{k} \;
	\mathbf{\boldsymbol{\mathcal{A}}}(\mathbf{k}, \omega) e^{j(\omega t - \mathbf{k}\cdot\mathbf{r})}
	\right]
	\end{align}
	\label{eq:k-spectralDefinition}
\end{subequations}
Since $\boldsymbol{\mathcal{A}}(\mathbf{r}, t)$ is a real function, one can drop the real part operator (cf previously. Still valid with causality?) which leads to the appearance of a four dimensional Fourier transform which pair is defined by:
 \begin{subequations}
 	\begin{align}
 	\boldsymbol{\mathcal{A}}(\mathbf{k}, \omega) =& 
 	\frac{1}{(2\pi)^4}
 	\int \diff t \int \diff \mathbf{r} \;
 	\mathbf{\boldsymbol{\mathcal{A}}}(\mathbf{r}, t) e^{-j(\omega t - \mathbf{k}\cdot\mathbf{r})}
 	\end{align}
 \end{subequations}

Using (\ref{eq:k-spectralDefinition}), the divergence and curl operators get simpler expressions in the $k-\omega$ domain\footnote{Note that the derivation requires an additional condition on the solution, which is to be 0 at infinity. Indeed, an integration per part is used during the evaluation of the nabla operator.} :
\begin{eqnarray}
\boldsymbol{\nabla} \cdot\boldsymbol{\mathcal{A}}(\mathbf{r}, t) 
&\leftrightarrow&
 -j\mathbf{k}\cdot \boldsymbol{\mathcal{A}} (\mathbf{k}, \omega)
\\
\boldsymbol{\nabla} \times \boldsymbol{\mathcal{A}} (\mathbf{r}, t) 
&\leftrightarrow& 
-j\mathbf{k}\times \boldsymbol{\mathcal{A}} (\mathbf{k}, \omega)
\\
\frac{\partial}{\partial t} \boldsymbol{\mathcal{A}} (\mathbf{r}, t) 
&\leftrightarrow& 
j \omega \boldsymbol{\mathcal{A}} (\mathbf{k}, \omega)
\end{eqnarray}
Plugging such a solution into the Maxwell equations (\ref{eq:MaxwellEquations}) leads to algebraic equations:
\begin{subequations}
	\begin{align}
	\mathbf{k} \times \boldsymbol{\mathcal{E}} (\mathbf{k}, \omega) 
		=& 
		\omega \boldsymbol{\mathcal{B}} (\mathbf{k}, \omega)
	\\
	\mathbf{k} \times \boldsymbol{\mathcal{H}} (\mathbf{k}, \omega) 
	=& 
	-\omega \boldsymbol{\mathcal{D}} (\mathbf{k}, \omega)
	+ 	
	j\boldsymbol{\mathcal{J}} (\mathbf{k}, \omega) 
	\\
	\mathbf{k}  \cdot \boldsymbol{\mathcal{D}} (\mathbf{k}, \omega) 
		=& jq_v(\mathbf{k}, \omega) 
	\\
	\mathbf{k}  \cdot \boldsymbol{\mathcal{B}} (\mathbf{k}, \omega) 
		=& 0
	\end{align}
	\label{eq:k-omegaMaxwellEquations}
\end{subequations}
Not only the equations are simpler to solve, but we will see in the section devoted to the constitutive relations that in some case, medium properties are also simpler in this domain.  


In the case of single plane-wave solution, such as:
\begin{equation}
\boldsymbol{\mathcal{E}}(\mathbf{r},t)
=
\Re\left[
\mathbf{E}_0
 e^{j(\omega_0 t - \mathbf{k}_0\cdot\mathbf{r})}
\right]
\end{equation}
then we obtain:
\begin{equation}
	\boldsymbol{\mathcal{E}}(\mathbf{k},\omega)
	=
	 	\frac{1}{(2\pi)^4}
	 	\mathbf{E}_0
	 	\delta(\omega - \omega_0) \delta(\mathbf{k} - \mathbf{k}_0) 
\end{equation}
and the Maxwell equations (\ref{eq:MaxwellEquations}) become:
\begin{subequations}
	\begin{align}
	\mathbf{k} \times\mathbf{E}_0
	=& 
	\omega \mathbf{B}_0 
	\\
%	\mathbf{k} \times \mathbf{H} 
%	=& 
%	-\omega \mathbf{D} 
%	+ 	
%	j\mathbf{J} 
%	\\
%	\mathbf{k}  \cdot \mathbf{D}  
%	=& jq_v(\mathbf{k}, \omega) 
%	\\
	\mathbf{k} \cdot \mathbf{B}  
	=& 0
	\end{align}
	\label{eq:k-omegaMaxwellEquationsPhasor}
\end{subequations}
where $\mathbf{B}_0 = (\mathbf{k}\times\mathbf{E}_0)/\omega$. Without any other information on the constitutive relations, we stop here in the derivation.

% ###########################################################################
% ###########################################################################
\subsection{Constitutive Relations}
Fluxes densities ($\mathcal{D}$,$\mathcal{B}$) differ from field intensities ($\mathcal{E},\mathcal{H}$) inside the material  with regards to relative magnitude and direction. Flux densities can be interpreted  as a response of the medium to an applied excitation\footnote{If we recall the Gauss law $	Q = \oint \boldsymbol{\mathcal{D}} \cdot \diff S$, the flux $\boldsymbol{\mathcal{D}}$ depends on the charge inside the closed surface and doesn't depend on the material itself, but the field intensity does. 	}
. Such, the constitutive relationships can be written generally as:
\begin{subequations}
	\begin{align}
		\boldsymbol{\mathcal{D}} =& \boldsymbol{\mathcal{D}}(\boldsymbol{\mathcal{E}},\boldsymbol{\mathcal{H}}) \\
		\boldsymbol{\mathcal{B}} =& \boldsymbol{\mathcal{B}}(\boldsymbol{\mathcal{E}},\boldsymbol{\mathcal{H}}) \\
		\boldsymbol{\mathcal{J}} =& \boldsymbol{\mathcal{J}}(\boldsymbol{\mathcal{E}},\boldsymbol{\mathcal{H}})
	\end{align}
\end{subequations}
All of these relations hold only if the time rate of change of the electromagnetic field is small enough. Otherwise, one needs to extend the definition of linearity using linear differential relations\parencite{Harrington2001, Jackson1998}:
\begin{subequations}
	\begin{align}
	\boldsymbol{\mathcal{D}} &= \varepsilon \boldsymbol{\mathcal{E}} + \varepsilon_1 \frac{\partial \boldsymbol{\mathcal{E}}}{\partial t} + \varepsilon_2 \frac{\partial^2 \boldsymbol{\mathcal{E}}}{\partial t^2} + \ldots \\
	\boldsymbol{\mathcal{B}} &= \varepsilon \boldsymbol{\mathcal{H}} + \varepsilon_1 \frac{\partial \boldsymbol{\mathcal{H}}}{\partial t} + \varepsilon_2 \frac{\partial^2 \boldsymbol{\mathcal{H}}}{\partial t^2} + \ldots
	\end{align}
\end{subequations}
Such situation arises typically when high intensity RF fields are used, which leads to non-linear phenomenons such \emph{ponderomotive effect}\parencite{Krapchev1979}.


%% ###########################################################################
%\subsubsection{General linear medium}
%\begin{subequations}
%	\begin{align}
%	\mathbf{D} =& \boldsymbol{\varepsilon}\mathbf{E} + \mathbf{P} \\
%	\mathbf{B} =& \boldsymbol{\mu}\left( \mathbf{H} + \mathbf{M} \right)
%	\end{align}
%\end{subequations} 
%where:
%\begin{itemize}
%	\item $\mathbf{P}$ is the electric polarization of the medium, caused by displacement of bounds charges, measured in $C/m^2$.  
%	\item $\mathbf{M}$ is the magnetisation of the medium (or magnetic polarization), which corresponds to the distribution of magnetic moments per unit volume, measured in $A/m$\footnote{Note that $\mathbf{M}$ is included in the parenthesis, while $\mathbf{P}$ is not, for a matter of historical definition. This of course affects the units of these quantities. }. 
%\end{itemize}

% ###########################################################################
\subsubsection{Vacuum}
In vacuum or in any other medium having similar characteristics than vacuum (such as air), the constitutive relationships have their most simpler form:
\begin{subequations}
 \begin{align}
  \boldsymbol{\mathcal{D}} &= \varepsilon_0 \boldsymbol{\mathcal{E}} \\
  \boldsymbol{\mathcal{B}} &= \mu_0 \boldsymbol{\mathcal{H}} \\
  \boldsymbol{\mathcal{J}} &= 0
 \end{align}
\end{subequations}
where $\varepsilon_0$ is the vacuum \emph{permittivity} and $\mu_0$ the vacuum \emph{permeability}. 

% ###########################################################################
\subsubsection{Isotropic linear mediums}
In a standard isotropic linear mediums, the constitutive relationships becomes linear relationships:
\begin{subequations}
 \begin{align}
  \boldsymbol{\mathcal{D}} &= \varepsilon \boldsymbol{\mathcal{E}} \\
  \boldsymbol{\mathcal{B}} &= \mu \boldsymbol{\mathcal{H}} \\
  \boldsymbol{\mathcal{J}} &= \sigma \boldsymbol{\mathcal{E}}
 \end{align}
\end{subequations}
where $\varepsilon$ and $\mu$ are the medium permittivity and permeability respectively. The parameter $\sigma$ is called the \emph{conductivity} of the medium. Note that these relationships generally do not hold when the field intensities are very large or in time varying medium.

Simple matter medium can be classified according its values of $\varepsilon, \mu$ and $\sigma$. Materials with high conductivity value $\sigma$ are called \emph{conductors} while those having a small value are referred as \emph{dielectrics} or \emph{insulators}. In electromagnetic models, good conductors are often approximated to \emph{perfect conductors}, characterized by the limit $\sigma\to\infty$. On the other hand, \emph{perfect dielectrics} assume $\sigma=0$. 

The medium permittivity $\varepsilon$ can never be less than the vacuum permittivity $\varepsilon_0$. The \emph{relative permittivity} is defined such as $\varepsilon_r=\varepsilon/\varepsilon_0$. The permittivity of a conductor is hard to measure but appears to be unity\parencite{Harrington2001}. A similar definition holds for the \emph{relative permeability} $\mu_r=\mu/\mu_0$. For almost all materials except \emph{ferromagnetic} materials, one has $\mu=\mu_0$.

% ###########################################################################
\subsubsection{Anisotropic mediums}
If the response of the medium is different depending on the direction of the oscillating field, then the medium is called \emph{anisotropic}. In this case, his response is expressed by tensor relationships.   In anisotropic linear mediums, the constitutive relationships becomes tensor-relationships:
\begin{subequations}
 \begin{align}
  \boldsymbol{\mathcal{D}} &= \boldsymbol{\varepsilon} \cdot \boldsymbol{\mathcal{E}} \\
  \boldsymbol{\mathcal{B}} &= \boldsymbol{\mu} \cdot  \boldsymbol{\mathcal{H}} \\
  \boldsymbol{\mathcal{J}} &= \boldsymbol{\sigma} \cdot  \boldsymbol{\mathcal{E}}
 \end{align}
\end{subequations}
where $\boldsymbol{\varepsilon}$, $\boldsymbol{\mu}$ and $\boldsymbol{\sigma}$ are the dielectric tensor, the permeability tensor and the conductivity tensor respectively, which can be interpreted as 3x3 matrices\parencite{Swanson2003}.  

% ###########################################################################
\subsubsection{Nonlocal medium}
If a medium exhibits a time or space dependence to an electromagnetic excitation, it is said to be \emph{nonlocal} or \emph{dispersive}, with respect to time and space respectively. In a time and is called \emph{time dispersive} medium, explicit time dependence arises as a time delay between the imposition of the electric field and the resulting polarization of the medium. This delay is due to the inertia of charged particles to respond to the time-varying field\parencite{Mackay2010, Brambilla1998}. In a \emph{space-dispersive} medium, the response at the location $\mathbf{r}$ and time $t$ can not only depends on the field at location $\mathbf{r}$ and time $t$, but of the field in its vicinity $\mathbf{r}'$ and by all previous instant $t'$. Spatial non-locality can be significant when the wavelength is comparable to some characteristic length–scale in the medium.  In plasma, the thermal agitation of the species induces add an additional erratic motion to the particles trajectory. Thus, the particles are influenced by the field in the domain explored by their motion\parencite{Brambilla1998}. This space dispersion can be omitted in the limit at which the temperature effects can be neglected. 

Thus, the constitutive relations of a general non-local anisotropic linear medium should be stated as:
\begin{subequations}
	\begin{align}
	\boldsymbol{\mathcal{D}}(\mathbf{r}, t) 
	= &
	\int_{t'=-\infty}^t \diff t'
	\int \diff \mathbf{r}' \;
	\boldsymbol{\varepsilon}(\mathbf{r},\mathbf{r}', t,t') \cdot \boldsymbol{\mathcal{E}}(\mathbf{r}', t') 
	\\
	\boldsymbol{\mathcal{B}}(\mathbf{r}, t)
	=& 
	\int_{t'=-\infty}^t \diff t'
	\int \diff \mathbf{r}' \;
	\boldsymbol{\mu}(\mathbf{r},\mathbf{r}', t,t') \cdot \boldsymbol{\mathcal{H}}(\mathbf{r}', t')  
	\\
	\boldsymbol{\mathcal{J}}(\mathbf{r}, t)
	=& 
	\int_{t'=-\infty}^t \diff t'
	\int \diff \mathbf{r}' \;
	\boldsymbol{\sigma}(\mathbf{r},\mathbf{r}', t,t') \cdot \boldsymbol{\mathcal{E}}(\mathbf{r}', t')  
	\end{align}
\end{subequations}
The restriction of time integration to times $t'<t$ is the expression of the causality, which impose that the quantities at $t$ can only be influenced by the quantities are previous instants.

If invariance with respect to the choice of origin in space (i.e. homogeneous) and time (i.e. stationary) can be asserted, then the previous relation can be expressed in terms of convolutions in the form of\parencite[p.19]{Brambilla1998}:
\begin{subequations}
	\begin{align}
	\boldsymbol{\mathcal{D}}(\mathbf{r}, t) 
	= &
	\int_{t'=-\infty}^t \diff t'
	\int \diff \mathbf{r}' \;
	\boldsymbol{\varepsilon}(\mathbf{r}-\mathbf{r}', t-t') \cdot \boldsymbol{\mathcal{E}}(\mathbf{r}', t') 
	\\
	\boldsymbol{\mathcal{B}}(\mathbf{r}, t)
	=& 
	\int_{t'=-\infty}^t \diff t'
	\int \diff \mathbf{r}' \;
	\boldsymbol{\mu}(\mathbf{r}-\mathbf{r}', t-t') \cdot \boldsymbol{\mathcal{H}}(\mathbf{r}', t')  
	\\
	\boldsymbol{\mathcal{J}}(\mathbf{r}, t)
	=& 
	\int_{t'=-\infty}^t \diff t'
	\int \diff \mathbf{r}' \;
	\boldsymbol{\sigma}(\mathbf{r}-\mathbf{r}', t-t') \cdot \boldsymbol{\mathcal{E}}(\mathbf{r}', t')  
	\end{align}
\end{subequations}


If one considers that the fields and current can be represented by a continuous spectrum of time-harmonic plane-waves such as done in section \ref{sec:spectralRepresentation}, which has the appearance of a four-dimensional Fourier transform \parencite{Clemmow1996}, i.e.:
\begin{subequations}
	\begin{align}
	\boldsymbol{\mathcal{E}}(\mathbf{r}, t) =& \Re \left[
	 \int \diff \omega \int \diff \mathbf{k} \;
	\boldsymbol{\mathcal{E}}(\mathbf{k}, \omega) e^{j(\omega t - \mathbf{k}\cdot\mathbf{r})}
	\right]
	\\
	\boldsymbol{\mathcal{H}}(\mathbf{r}, t) =& \Re \left[
	 \int \diff \omega \int \diff \mathbf{k} \;
	\boldsymbol{\mathcal{H}}(\mathbf{k}, \omega) e^{j(\omega t - \mathbf{k}\cdot\mathbf{r})}
	\right]
	\\
	\boldsymbol{\mathcal{J}}(\mathbf{r}, t) =& \Re \left[ \int \diff \omega \int \diff \mathbf{k} \;
	\boldsymbol{\mathcal{J}}(\mathbf{k}, \omega) e^{j(\omega t - \mathbf{k}\cdot\mathbf{r})} \right]
	\end{align}
\end{subequations}
In such as case, replacing the fields with the latter definition leads in the $k-\omega$ domain, to simpler algebraic time-invariant relationships thanks to the convolution theorem:
\begin{subequations}
	\begin{align}
	\boldsymbol{\mathcal{D}}(\mathbf{k}, \omega) 
	=& 
	\boldsymbol{\varepsilon}(\mathbf{k}, \omega) \cdot \boldsymbol{\mathcal{E}}(\mathbf{k}, \omega) 
	\\
	\boldsymbol{\mathcal{B}}(\mathbf{k}, \omega) 
	=& 
	\boldsymbol{\mu}(\mathbf{k}, \omega) \cdot \boldsymbol{\mathcal{H}}(\mathbf{k}, \omega) 
	\\
	\boldsymbol{\mathcal{J}}(\mathbf{k}, \omega) 
	=& 
	\boldsymbol{\sigma}(\mathbf{k}, \omega) \cdot \boldsymbol{\mathcal{E}}(\mathbf{k}, \omega) 
	\end{align}
	\label{eq:k-omegaDispersionRelation}
\end{subequations}
where the tensors have been defined by:
\begin{subequations}
	\begin{align}
	\boldsymbol{\sigma}(\mathbf{k}, \omega) 	
	=
	\frac{1}{(2\pi)^4}
	\int_0^\infty \diff t \int \diff \mathbf{r} \;
	\boldsymbol{\sigma}(\mathbf{r}, t) 	e^{-j(\omega t - \mathbf{k}\cdot\mathbf{r})}
	\end{align}
\end{subequations}




% ###########################################################################
% ###########################################################################
\subsection{Boundary conditions}

% ###########################################################################
% ###########################################################################
% ###########################################################################
\section{Transmission Line Theory}

% ###########################################################################
% ###########################################################################
% ###########################################################################
\section{Microwave safety}
\parencite[sec.5.8.3]{Benford2015}



\chapter{Dispersion Relation}
\section{Vacuum Dispersion Relation}
In a non-dispersive isotropic homogeneous non-lossy dielectric medium, such as vacuum (or air), the wavevector $\mathbf{k}=k\mathbf{\hat{k}}$ direction is given by:
\begin{equation}
 \mathbf{\hat{k}}\cdot\mathbf{E}=0
 \;\;\;\;\;\;
 \mathbf{H}=\frac{n}{Z_{0}}\hat{\mathbf{k}}\times\mathbf{E}\label{eq:vecteur_onde_vide}
\end{equation}
where $\mathbf{\hat{k}}$ the unit vector in the direction of $\mathbf{k}$, $k=|\mathbf{k}|=n k_{0}$ is the wavenumber and $n=\sqrt{\varepsilon_{r}}$ the medium optical index. $Z_{0}=\sqrt{\frac{\mu_{0}}{\varepsilon_{0}}}=\mu_{0}c_{0}=\frac{1}{\varepsilon_{0}c_{0}}$ is the \emph{vacuum impedance}. The wave equation in such a medium is:
\begin{equation}
 \mathbf{k} \times \mathbf{k} \times \mathbf{E} + k^{2} \mathbf{E} = \mathbf{0}
\end{equation}

The condition for that system of 3 equations and 3 unknowns $(E_{x},E_{y},E_{z})$ to have a non-trivial solution, consists in solving the \emph{dispersion relation} between the wavevector $\mathbf{k}$ and the angular frequency $\omega$, i.e. $\mathbf{k}(\omega)$. In this medium, this relation is simple (TODO demo): $k=\sqrt{\mu\varepsilon}\omega=\frac{c}{n}\omega$\parencite[(7.4)]{Jackson1998}\footnote{While in a lossy medium $k^{2}=\mu\varepsilon\omega^{2}-j\omega\mu\sigma$ \parencite[sec. 8.2]{Bladel2007}.}.

\subsection{Cold Plasma Dispersion Relation}
Starting from Maxwell equations expressed in the $k-\omega$ domain (\ref{eq:k-omegaMaxwellEquations}) and assuming that the dispersion relations are of the form (\ref{eq:k-omegaDispersionRelation}), with the supplementary assumption that the medium is non magnetic (i.e. $\boldsymbol{\mu}=\mu_0$), one has thus:
\begin{subequations}
	\begin{align}
	\mathbf{k} \times \boldsymbol{\mathcal{E}} (\mathbf{k}, \omega) 
	=& 
	\omega \mu_0 \boldsymbol{\mathcal{H}}(\mathbf{k}, \omega) 
	\\
	\mathbf{k} \times \boldsymbol{\mathcal{H}} (\mathbf{k}, \omega) 
	=& 
	\left(
	-\omega \boldsymbol{\varepsilon} (\mathbf{k}, \omega) 
	+ 	
	j\boldsymbol{\sigma} (\mathbf{k}, \omega) 
	\right) \cdot 
	\boldsymbol{\mathcal{E}}(\mathbf{k}, \omega)
	\\
	\mathbf{k}  \cdot \boldsymbol{\varepsilon}(\mathbf{k}, \omega)  \cdot \boldsymbol{\mathcal{E}} (\mathbf{k}, \omega) 
	=& jq_v(\mathbf{k}, \omega) 
	\\
	\mathbf{k}  \cdot \boldsymbol{\mathcal{H}} (\mathbf{k}, \omega) 
	=& 0	
	\end{align}
\end{subequations}
combining the first two equations leads to the propagation equation:
\begin{equation}
\mathbf{k} \times \mathbf{k} \times \boldsymbol{\mathcal{E}} (\mathbf{k}, \omega) 
-
	\left(
	-\omega^2 \mu_0 \boldsymbol{\varepsilon} (\mathbf{k}, \omega) 
	+ 	
	j\omega\mu_0\boldsymbol{\sigma} (\mathbf{k}, \omega) 
	\right) \cdot 
	\boldsymbol{\mathcal{E}}(\mathbf{k}, \omega)
	=
	\boldsymbol{0}
\end{equation}
Let's define the equivalent relative dielectric tensor $\mathbf{K}$ such as:
\begin{equation}
\mathbf{K}(\mathbf{k},\omega)
=
\frac{j}{\omega\varepsilon_0}\boldsymbol{\sigma} (\mathbf{k}, \omega)
-\mathbf{I}
\end{equation}
then,
\begin{equation}
\mathbf{k} \times \mathbf{k} \times \boldsymbol{\mathcal{E}} (\mathbf{k}, \omega) 
- k_0^2
\mathbf{K}(\mathbf{k}, \omega) \cdot 
\boldsymbol{\mathcal{E}}(\mathbf{k}, \omega)
=
\boldsymbol{0}
\end{equation}
where $k_0=\frac{\omega}{c}$
% Dans un plasma froid magn�tis�, la situation est radicalement diff�rente,
% car les courants de polarisation g�n�r�s par les mouvements �lectroniques
% et ioniques modifient la polarisation des ondes planes ainsi que leur
% dispersion. Diff�rentes branches de dispersion, ou \emph{modes}, apparaissent\cite[chap.8]{Rax2005}. 
% 
% On rappelle l'expression des �quations de Maxwell en r�gime harmonique
% pour une onde plane de vecteur d'onde $\mathbf{k}$ :
% 
% \begin{eqnarray}
% \mathbf{k}\times\mathbf{E} & = & \omega\mu_{0}\mathbf{H}\\
% \mathbf{k}\times\mathbf{H} & = & -\omega\varepsilon_{0}\mathbf{K}\cdot\mathbf{E}
% \end{eqnarray}
% Pour d�terminer les propri�t�s de ces modes, on �tudie les solutions
% de l'\emph{�quation d'onde} d�duite des deux pr�c�dentes �quations\footnote{NB : L'�quation d'onde ne d�pend pas de la convention temporelle choisie.}
% :
% 
% \begin{equation}
% \mathbf{n}\times\mathbf{n}\times\mathbf{\tilde{E}}+\mathbb{K}\cdot\mathbf{\tilde{E}}=\mathbf{0}\label{eq:Helmoltz}
% \end{equation}
% o� $\mathbf{n}=\mathbf{k}/k_{0}$ correspond au vecteur d'indice de
% r�fraction, dont la direction est celle du vecteur d'onde $\mathbf{k}$
% et l'amplitude celle de l'indice de r�fraction. A priori, la matrice
% $\mathbb{K}$ d�pend des trois composantes du vecteur d'onde $\mathbf{k}$.
% En choisissant le syst�me de coordonn�es cart�sien l'�quation \ref{eq:Helmoltz}
% peut s'�crire sous forme matricielle\footnote{On peut s'�conomiser un peu de calcul vectoriel gr�ce � un peu d'alg�bre
% et le formalisme des dyadiques\cite{Belov2003,Lindell1995}. En utilisant
% l'identit� ``BAC-CAB'', le double produit vectoriel s'�crit $\mathbf{n}\times\mathbf{n}\times\mathbf{\tilde{E}}=\mathbf{n}\left(\mathbf{n}\cdot\mathbf{\tilde{E}}\right)-\mathbf{\tilde{E}}\left(\mathbf{n}\cdot\mathbf{n}\right)=\mathbf{n}\left(\mathbf{n}\cdot\mathbf{\tilde{E}}\right)-n^{2}\mathbf{\tilde{E}}$.
% Avec l'op�ration dyadique $\mathbf{n}\left(\mathbf{n}\cdot\tilde{\mathbf{E}}\right)=\mathbf{nn}\cdot\mathbf{\tilde{E}}$
% et $\mathbb{I}=\mathbf{xx}+\mathbf{yy}=\mathbf{zz}$ l'op�rateur dyadique
% unit� v�rifiant $\mathbf{\tilde{E}}=\mathbb{I}\cdot\tilde{\mathbf{E}}$on
% peut alors factoriser l'�quation \ref{eq:Helmoltz} en un op�rateur
% dyadique $\left(\mathbf{nn}-n^{2}\mathbb{I}+\mathbb{K}\right)\cdot\tilde{\mathbf{E}}=\mathbf{0}$
% qui donne directement l'expression matricielle de l'�quation \ref{eq:relation_disp_matr}.} :
% \begin{equation}
% \left(\begin{array}{ccc}
% K_{xx}-n_{y}^{2}-n_{z}^{2} & K_{xy}+n_{x}n_{y} & K_{xz}+n_{x}n_{z}\\
% K_{yx}+n_{x}n_{y} & K_{yy}-n_{x}^{2}-n_{z}^{2} & K_{yz}+n_{y}n_{z}\\
% K_{zx}+n_{x}n_{z} & K_{zy}+n_{y}n_{z} & K_{zz}-n_{x}^{2}-n_{y}^{2}
% \end{array}\right)\left(\begin{array}{c}
% \tilde{E_{x}}\\
% \tilde{E}_{y}\\
% \tilde{E}_{z}
% \end{array}\right)=\mathbf{0}\label{eq:relation_disp_matr}
% \end{equation}
%  Si l'on suppose en plus que le vecteur d'onde $\mathbf{k}$ (ie.
% la propagation de l'onde) soit contenu dans le plan $x-z$ (ie $k_{y}=n_{y}=0$),
% l'�quation pr�c�dente devient :
% 
% \begin{equation}
% \left(\begin{array}{ccc}
% K_{xx}-n_{z}^{2} & K_{xy} & K_{xz}+n_{x}n_{z}\\
% K_{yx} & K_{yy}-n_{x}^{2}-n_{z}^{2} & K_{yz}\\
% K_{zx}+n_{x}n_{z} & K_{zy} & K_{zz}-n_{x}^{2}
% \end{array}\right)\left(\begin{array}{c}
% \tilde{E_{x}}\\
% \tilde{E}_{y}\\
% \tilde{E}_{z}
% \end{array}\right)=\mathbf{0}\label{eq:relation_disp_matr_full}
% \end{equation}
% Si le plasma est homog�ne (ind�pendant de $\mathbf{r}$) on peut exploiter
% l'�quivalence entre toutes les directions perpendiculaires au champ
% magn�tique statique pour pr�dire que $\mathbb{K}$ doit �tre fonction
% de $k_{\parallel}$et $k_{\perp}^{2}$ seulement. 
% 
% \begin{figure}
% \begin{centering}
% \includegraphics[width=0.5\textwidth]{figures/geometrie_dispersion_plasma}
% \par\end{centering}
% \caption{G�om�trie cart�sienne du milieu plasma.\label{fig:G=0000E9om=0000E9trie-cart=0000E9sienne-plasma}}
% \end{figure}
% 
% Si $\mathbb{K}$ est le tenseur de permittivit� d'un plasma froid
% (\ref{eq:tenseur_stix}) d�finit dans l'annexe \ref{sec:Tenseur-de-permittivit=0000E9},
% c'est-�-dire tel que $z$ soit parall�le au champ magn�tique (Figure
% \ref{fig:G=0000E9om=0000E9trie-cart=0000E9sienne-plasma}), alors
% on a:
% 
% \begin{equation}
% \left(\begin{array}{ccc}
% S-n_{z}^{2} & jD & n_{x}n_{z}\\
% -jD & S-n_{x}^{2}-n_{z}^{2} & 0\\
% n_{x}n_{z} & 0 & P-n_{x}^{2}
% \end{array}\right)\left(\begin{array}{c}
% \tilde{E_{x}}\\
% \tilde{E}_{y}\\
% \tilde{E}_{z}
% \end{array}\right)=\mathbf{0}\label{eq:relation_disp_matr_froid}
% \end{equation}
% On d�finit $\theta$ comme l'angle entre le vecteur d'onde $\mathbf{k}$
% et la direction $\hat{\mathbf{e}}_{z}$ du champ magn�tique, soit
% $n_{x}=n_{\perp}=n\sin\theta$ et $n_{z}=n_{\parallel}=n\cos\theta$,
% on a alors \footnote{Pour obtenir la version harmonique en convention $-j\omega t$, il
% faut remplacer $n^{2}\rightarrow-n^{2}$ et $j\rightarrow-j$.}:
% 
% \begin{equation}
% \left(\begin{array}{ccc}
% S-n^{2}\cos^{2}\theta & jD & n^{2}\cos\theta\sin\theta\\
% -jD & S-n^{2} & 0\\
% n^{2}\cos\theta\sin\theta & 0 & P-n^{2}\sin^{2}\theta
% \end{array}\right)\left(\begin{array}{c}
% \tilde{E}_{x}\\
% \tilde{E}_{y}\\
% \tilde{E}_{z}
% \end{array}\right)=\mathbf{0}\label{eq:cold_plasma_dispersion_relation_matrix}
% \end{equation}
% 
% L'existence de solutions non triviales � l'�quation d'onde (les modes)
% (\ref{eq:relation_disp_matr_full}) n�cessite que le d�terminant de
% la matrice soit nul. Cette condition donne la \emph{relation de dispersion},
% qui pour un plasma froid peut s'�crire \cite[p.8-9]{Stix1992}\cite[�2.1.3]{Swanson2003}\cite[�18.1]{Brambilla1998}:
% 
% \begin{equation}
% \boxed{An^{4}-Bn^{2}+C=0}\label{eq:cold_plasma_dispersion_relation_n}
% \end{equation}
% avec\footnote{O� l'on a remarqu� que $S^{2}-D^{2}=RL$. Les expressions $A,B,C$
% ne d�pendent pas de la convention temporelle choisie. }:
% 
% \begin{eqnarray}
% A & = & S\sin^{2}\theta+P\cos^{2}\theta\\
% B & = & \left(S^{2}-D^{2}\right)\sin^{2}\theta+PS\left(1+\cos^{2}\theta\right)=RL\sin^{2}\theta+PS\left(1+\cos^{2}\theta\right)\\
% C & = & P\left(S^{2}-D^{2}\right)=PRL
% \end{eqnarray}
% Soit $n=n(\theta,\omega)=n(\hat{\mathbf{k}},\omega)$ la solution
% de la relation de dispersion pour une fr�quence $\omega$ et une direction
% de propagation $\hat{\mathbf{k}}$ (ie. $\theta$) donn�es. Une onde
% plane d'indice $n$ et de nombre d'onde $\mathbf{k}=n\frac{\omega}{c}\mathbf{\hat{k}}$
% peut se propager dans le plasma � la fr�quence $\omega$ et dans la
% direction du vecteur unitaire $\mathbf{\hat{k}}$ en l'absence de
% sources ext�rieures (plus exactement avec des sources situ�es � l'infini).
% Une telle onde est appel�e \emph{onde caract�ristique} ou \emph{mode
% de propagation} (mode propre) du plasma.
% 
% L'�quation (\ref{eq:cold_plasma_dispersion_relation_n}) est une �quation
% du second degr� en $n^{2}$ ayant pour solution : 
% \begin{equation}
% n^{2}=\frac{B\pm\sqrt{\Delta}}{2A}\label{eq:solution_cold_plasma_dispersion_relation_n}
% \end{equation}
% o� son d�terminant $\Delta=B^{2}-4AC$ vaut : 
% \begin{equation}
% \Delta=(RL-PS)^{2}\sin^{4}\theta+4P^{2}D^{2}\cos^{2}\theta\label{eq:cold_plasma_determinant_dispersion_relation_n}
% \end{equation}
% 
% Le d�terminant (\ref{eq:cold_plasma_determinant_dispersion_relation_n})
% n'est jamais n�gatif, ce qui signifie que l'�quation (\ref{eq:cold_plasma_dispersion_relation_n})
% a toujours deux solutions r�elles et distinctes en $n^{2}$. Ainsi,
% les ondes planes dans un plasma froid sont soit purement propagatives
% ($n^{2}>0$) soit purement �vanescentes ($n^{2}<0$) ; les oscillations
% amorties sont exclues. La transition entre ces deux r�gimes a lieu
% aux coupures ($n=0$) et aux r�sonances ($n\to\infty$). 
% 
% Les deux racines peuvent �tre confondues lorsque le d�terminant �tre
% nul, dans les cas particulier suivant :
% \begin{itemize}
% \item En propagation parall�le, ie $\theta=0$, lorsque $P=0$ ;
% \item En propagation perpendiculaire, ie $\theta=\pi/2$, lorsque $RL=PS$.
% \end{itemize}
% Dans le cadre de l'approximation froide d�finie par l\textquoteright absence
% de \emph{dispersion spatiale}, c'est-�-dire lorsque les �l�ments du
% tenseur $\mathbb{K}$ ne d�pendent pas de l'indice de r�fraction $\mathbf{n}$
% (ie. du vecteur d'onde $\mathbf{k}$), l'�quation (\ref{eq:cold_plasma_dispersion_relation_n})
% est une simple �quation quadratique en $n^{2}$. Il existe donc deux
% solutions distinctes qui peuvent se propager dans le plasma ; un plasma
% froid est donc un milieu \emph{bir�fringent} (les ondes peuvent �tre
% �vanescentes ou non, selon les caract�ristiques du plasma). Si on
% avait introduit des effets thermiques, les �l�ments du tenseur de
% permittivit� d�pendraient alors du vecteur d'onde et de nouveaux modes
% apparaitraient\cite[�1.2.2]{Dumont2007}. 
% 
% \subsubsection{Expression en fonction de $\theta$.}
% 
% L'�quation de dispersion (\ref{eq:cold_plasma_dispersion_relation_n})
% peut �tre exprim�e sous diverses formes �quivalentes. La relation
% de dispersion (\ref{eq:cold_plasma_dispersion_relation_n}) peut �tre
% exprim�e en fonction de l'angle $\theta$\cite[�18.2]{Brambilla1998}:
% 
% \begin{equation}
% \tan^{2}\theta=-\frac{P\left(n^{2}-R\right)\left(n^{2}-L\right)}{\left(Sn^{2}-RL\right)\left(n^{2}-P\right)}\label{eq:cold_plasma_dispersion_relation_theta}
% \end{equation}
% En r�solvant pour $n^{2}$ on obtient un cas particulier des �quations
% d'Appleton-Hartree, et en particulier pour $D=0$
% \begin{equation}
% n^{2}=\begin{cases}
% \frac{PS}{S\sin^{2}\theta+P\cos^{2}\theta} & \mbox{extraordinary mode}\\
% S & \mbox{oridnary mode}
% \end{cases}\label{eq:cold_plasma_dispersion_relation_solution_n}
% \end{equation}
% 
% 
% \subsubsection{Expression en fonction de $n_{\parallel}$.}
% 
% Lorsque le nombre d'onde $n_{\parallel}=n_{z}$ est d�finit par des
% conditions ext�rieures, comme la structure d'une antenne, et en supposant
% que la propagation est contrainte au plan (xOz) ($n_{y}=0$), la relation
% de dispersion peut �tre exprim�e en fonction de $n_{\perp}^{2}=n_{x}^{2}=n^{2}-n_{\parallel}^{2}$.
% Ainsi, le d�terminant de (\ref{eq:relation_disp_matr_froid}) donne
% une �quations quadratique en $n_{\perp}^{2}$\cite[�18.2]{Brambilla1998}\footnote{En convention $-j\omega t$: $A_{1}n_{\perp}^{4}-B_{1}n_{\perp}^{2}+C_{1}=0$ }:
% 
% \begin{equation}
% A_{1}n_{\perp}^{4}+B_{1}n_{\perp}^{2}+C_{1}=0\label{eq:cold_plasma_dispersion_relation_n_perp}
% \end{equation}
% avec 
% 
% \begin{eqnarray}
% A_{1} & = & S\\
% B_{1} & = & \left(P+S\right)\left(n_{\parallel}^{2}-S\right)+D^{2}=RL+PS-n_{\parallel}^{2}(P+S)\\
% C_{1} & = & P\left(\left(n_{\parallel}^{2}-S\right)^{2}-D^{2}\right)=P(n_{\parallel}^{2}-R)(n_{\parallel}^{2}-L)
% \end{eqnarray}
% o� on rappelle que $R=S+D$ et $L=S-D$. Les coupures, qui correspondent
% aux transitions entre propagation et �vanescence, sont alors d�finies
% pour $n_{\perp}=0$. Les r�sonances pour $n_{\perp}\to\infty$. Comme
% pr�c�demment, le d�terminant de l'�quation est toujours positif ou
% nul, l'�quation (\ref{eq:cold_plasma_dispersion_relation_n_perp})
% poss�de donc deux solutions, qui peuvent �ventuellement �tre complexes
% conjugu�es selon les conditions (fr�quence, densit�, etc.). La solution
% s'exprime par \cite[�15.9.2]{Friedberg2007}:
% 
% \begin{equation}
% n_{\perp}^{2}=-\frac{P}{2S}\left(D^{2}/P-S-n_{\parallel}^{2}\pm\sqrt{\left(D^{2}/P-S-n_{\parallel}^{2}\right)^{2}+\frac{4SD^{2}}{P}}\right)\label{eq:cold_plasma_solution_n_perp}
% \end{equation}
% La solution dont la valeur est la plus grande, c'est-�-dire pour laquelle
% la valeur de la vitesse de phase perpendiculaire $v_{\perp}=\omega/k_{\perp}$
% sera la plus petite, correspond � la branche dite \emph{lente}, l'autre
% branche �tant la solution dite \emph{rapide}.
% 
% \begin{figure}[h]
% \begin{centering}
% \includegraphics[width=0.9\textwidth]{figures/n_perp_vs_ne}\label{Flo:n_perp_vs_ne}\caption{Exemple de solutions de l'�quation de dispersion pour $n_{\perp}^{2}$
% en fonction de la densit� �lectronique $n_{e}$. \textbf{$n_{\parallel}=2.0$
% }et\textbf{ $B_{0}=2.95$~}T, f=3.7GHz.}
% \par\end{centering}
% \end{figure}
% 
% Pour des valeurs de $\left|P\right|$ grande ($\omega\ll\omega_{pe}$),
% les deux racines (\ref{eq:cold_plasma_solution_n_perp}) peuvent s'approcher
% par au premier ordre en $\omega^{2}/\omega_{pe}^{2}$ (cf. \cite[p.222]{Brambilla1998}).
% De la m�me fa�on, dans le voisinage des fr�quences LH, on peut faire
% l'hypoth�se $D\approx0$. Dans ce cas, l'�quation de dispersion s'�crit
% (avec la convention temporelle $e^{j\omega t}$):
% \begin{eqnarray}
% A_{1}n_{\perp}^{4}+B_{1}n_{\perp}^{2}+C_{1} & = & 0\label{eq:cold_plasma_dispersion_relation_n_perp_approx1}
% \end{eqnarray}
% avec
% \begin{eqnarray}
% A_{1} & = & S\\
% B_{1} & = & S^{2}+PS-n_{\parallel}^{2}(P+S)\\
% C_{1} & = & P(n_{\parallel}^{2}-S)^{2}
% \end{eqnarray}
% Les solutions de l'�quation de dispersion \ref{eq:cold_plasma_dispersion_relation_n_perp_approx1}
% sont dans ce cas :
% \begin{eqnarray}
% n_{\perp}^{2}=n_{\perp,F}^{2} & = & -\left(S+n_{\parallel}^{2}\right)\label{eq:n_perp_fast_wave_solution_approximate}\\
% n_{\perp}^{2}=n_{\perp,S}^{2} & = & -\frac{P}{S}\left(S+n_{\parallel}^{2}\right)\label{eq:n_perp_slow_wave_solution_approximate}
% \end{eqnarray}
% Enfin, toujours au voisinage du domaine de plasma de bord, $S\approx1$
% d'o� \ref{eq:cold_plasma_dispersion_relation_n_perp_approx1} :
% \begin{equation}
% n_{\perp}^{4}+\left(1-n_{\parallel}^{2}\right)\left(1+P\right)n_{\perp}^{2}+P\left(n_{\parallel}^{2}-1\right)^{2}=0
% \end{equation}
% qui a pour solutions:
% \begin{eqnarray}
% n_{\perp}^{2}=n_{\perp,F}^{2} & = & -\left(1-n_{\parallel}^{2}\right)\label{eq:n_perp_fast_wave_solution_approximate2}\\
% n_{\perp}^{2}=n_{\perp,S}^{2} & = & -P\left(1-n_{\parallel}^{2}\right)\label{eq:n_perp_slow_wave_solution_approximate2}
% \end{eqnarray}
% 
% 
% \subsection{Polarisation des champs}
% 
% Chaque mode propre de la solution de dispersion poss�de une polarisation
% d�finie par la relation de dispersion exprim�e sous forme matricielle
% (\ref{eq:relation_disp_matr_froid}). Ainsi, on d�duit des deux derni�res
% relations les relations existantes entre les composantes $\tilde{E}_{x}$
% et $\tilde{E}_{y}$\footnote{En convention $-j\omega t$ : $\frac{\tilde{E}_{y}}{\tilde{E}_{x}}=-\frac{jD}{S-n^{2}}$
% et $\frac{\tilde{E}_{z}}{\tilde{E}_{x}}=-\frac{n_{\perp}n_{\parallel}}{P-n_{\perp}^{2}}$.}:
% 
% \begin{equation}
% \frac{\tilde{E}_{y}}{\tilde{E}_{x}}=\frac{jD}{S+n^{2}}\label{eq:relation_entre_composantes_X_Y}
% \end{equation}
% et entre les composantes $\tilde{E}_{x}$ et $\tilde{E}_{z}$:
% 
% \begin{equation}
% \frac{\tilde{E}_{z}}{\tilde{E}_{x}}=\frac{n_{\perp}n_{\parallel}}{P+n_{\perp}^{2}}\label{eq:relation_entre_composantes_X_Z}
% \end{equation}
% 
% En utilisant la relation issue de la premi�re ligne de (\ref{eq:relation_disp_matr_froid}),
% on en d�duit une expression g�n�rale pour le champ radial en fonction
% des composantes transverses y et z :
% \begin{equation}
% \tilde{E}_{x}=\frac{-jD\tilde{E}_{y}+n_{\perp}n_{\parallel}\tilde{E}_{z}}{S+n_{\parallel}^{2}}\label{eq:relation_entre_composantes_X_Y_Z}
% \end{equation}
% En injectant cette expression dans les deux pr�c�dentes, on trouve
% :
% 
% \begin{eqnarray}
% \left[\left(S+n_{\perp}^{2}+n_{\parallel}^{2}\right)\left(S+n_{\parallel}^{2}\right)-D^{2}\right]\tilde{E}_{y}-jDn_{\perp}n_{\parallel}\tilde{E}_{z} & = & 0\label{eq:relation_entre_composantes_Y_Z_1}\\
% jDn_{\perp}n_{\parallel}\tilde{E}_{y}+\left[\left(P+n_{\perp}^{2}\right)\left(S+n_{\parallel}^{2}\right)-n_{\perp}^{2}n_{\parallel}^{2}\right]\tilde{E}_{z} & = & 0\label{eq:relation_entre_composantes_Y_Z_2}
% \end{eqnarray}
% 
% 
% \subsubsection{Polarisation dominante des modes lents et rapides}
% 
% En prenant \ref{eq:relation_entre_composantes_Y_Z_1} pour $n_{\perp}^{2}=n_{\perp,F}^{2}$,
% on obtiens le condition de polarisation de l'onde rapide au premier
% ordre :
% 
% \begin{equation}
% \tilde{E}_{z}=0\label{eq:polarisation_FastWave}
% \end{equation}
% En prenant \ref{eq:relation_entre_composantes_Y_Z_2} pour $n_{\perp}^{2}=n_{\perp,S}^{2}$,
% on obtiens la condition de polarisation de l'onde lente au premier
% ordre :
% 
% \begin{equation}
% \tilde{E}_{y}=0\label{eq:polarisation_SlowWave}
% \end{equation}
% 
% 
% \subsubsection{Polarisation dans le plasma}
% 
% En prenant \ref{eq:relation_entre_composantes_X_Z} pour $n_{\perp}^{2}=n_{\perp,S}^{2}$
% dans l'hypoth�se ou $D\approx0$ et $P<0$, il vient
% \begin{equation}
% \frac{\tilde{E}_{z}}{\tilde{E}_{x}}=\frac{\left(-\frac{P}{S}\left(S+n_{\parallel}^{2}\right)\right)^{1/2}n_{\parallel}}{P-\frac{P}{S}\left(S+n_{\parallel}^{2}\right)}=\frac{\left(S\left(S+n_{\parallel}^{2}\right)\right)^{1/2}}{\left|P\right|^{1/2}n_{\parallel}}
% \end{equation}
% par cons�quent, � mesure que la densit� locale augmente ($\left|P\right|$
% augmente, et augmente plus vite que $S$), la polarisation dominante
% est radiale ($\tilde{E}_{x}$).
% 
% \subsection{Vitesse de phase}
% 
% \subsection{Vitesse de groupe}
% 
% La vitesse de groupe est donn�e par\cite[�7.2, �18.5]{Brambilla1998}:
% 
% \begin{equation}
% \mathbf{v}_{g}\left(\mathbf{k}_{0}\right)=\left.\frac{\partial\omega}{\partial\mathbf{k}}\right|_{\mathbf{k}=\mathbf{k}_{0}}
% \end{equation}
% ou, en posant Soit $\mathcal{D}$ la partie gauche de l'�quation \ref{eq:cold_plasma_dispersion_relation_n}:
% 
% \begin{equation}
% \mathbf{v}_{g}\left(\mathbf{k}_{0}\right)=-\left.{\displaystyle \frac{\frac{\partial\mathcal{D}}{\partial\mathbf{k}}}{\frac{\partial\mathcal{D}}{\partial\omega}}}\right|_{\mathbf{k}=\mathbf{k}_{0}}
% \end{equation}
% 
% Dans un milieu anisotrope comme le plasma froid, la vitesse de groupe,
% qui est la vitesse de propagation de l'�nergie, n'est g�n�ralement
% pas colin�aire avec le vecteur d'onde \textbf{$\mathbf{k}$}. Aussi,
% on d�compose g�narelement la vitesse de groupe selon ses composantes
% parall�les et perpendiculaires au vecteur d'onde. Soit
% \begin{equation}
% \mathbf{v}_{g}=v_{gr}\mathbf{\hat{k}}+v_{g\theta}\mathbf{\hat{e}}_{\theta}
% \end{equation}
% avec $\mathbf{\hat{k}}=\mathbf{k}/k_{0}$ et $\mathbf{\hat{e}}_{\theta}=\mathbf{\hat{k}}\times\left(\mathbf{B}_{0}\times\mathbf{\hat{k}}\right)/B_{0}$
% le vecteur unit� perpendiculaire � $\mathbf{\hat{k}}$ dans le plan
% form� par $\mathbf{B}_{0}$ et $\mathbf{\hat{k}}$, et
% 
% \begin{equation}
% v_{gr}=-\frac{1}{k_{0}}\frac{\partial\mathcal{D}/\partial n}{\partial\mathcal{D}/\partial\omega}\quad\quad v_{g\theta}=-\frac{1}{nk_{0}}\frac{\partial\mathcal{D}/\partial\theta}{\partial\mathcal{D}/\partial\omega}
% \end{equation}
% 
% Appliqu� � \ref{eq:cold_plasma_dispersion_relation_n}, on a :
% \begin{equation}
% \end{equation}
% 
% L'angle entre le vecteur d'onde $\mathbf{k}$ et la vitesse de groupe
% est alors 
% \begin{equation}
% \tan\alpha_{g}=\frac{v_{g\theta}}{v_{gr}}=\frac{1}{n}\frac{\partial\mathcal{D}/\partial\theta}{\partial\mathcal{D}/\partial n}=-\frac{1}{n}\frac{dn}{d\theta}
% \end{equation}
% 
% 
% \subsection{R�flexion/Transmission d'une onde plane par un plasma froid magn�tis�}
% 
% Soit une onde plane incidente en provenance d'un milieu de permittivit�
% $\varepsilon^{i}$ sur un plasma magn�tis� froid. Le champ incident
% a pour expression g�n�rale $\mathbf{E}^{i}e^{-j\mathbf{k}^{i}\cdot\mathbf{r}}$,
% le champ r�fl�chis $\mathbf{E}^{r}e^{-j\mathbf{k}^{r}\cdot\mathbf{r}}$
% et le champ transmis $\mathbf{E}^{t}e^{-j\mathbf{k}^{t}\cdot\mathbf{r}}$.
% Le vecteur d'onde de l'onde plane incidente $\mathbf{k}^{i}$ est
% contenu dans le plan $xOz$ et forme un angle $\theta^{i}$ avec l'axe
% $z$ (Figure \ref{fig:G=0000E9om=0000E9trie-reflexion}), i.e. $\mathbf{k}^{i}=\sqrt{\varepsilon^{i}}k_{0}\left(\sin\theta^{i}\mathbf{\hat{x}}+\cos\theta^{i}\mathbf{\hat{z}}\right)$.
% Supposons par ailleurs que le champ �lectrique \textbf{$\mathbf{E}^{i}$}
% soit �galement inclus dans ce m�me plan, ie. $\mathbf{E}^{i}=E^{i}\left(\cos\theta^{i}\mathbf{\hat{x}}-\sin\theta^{i}\mathbf{\hat{z}}\right)$
% (mode TM). Les conditions aux limites �tablissent que les composantes
% transverses du champ �lectrique doivent �tre continues � l'interface,
% soit\footnote{En toute g�n�ralit�, on doit avoir $\mathbf{E}_{t}^{i}+\mathbf{E}_{t}^{r}=\mathbf{E}_{t}^{t}$
% � $x=0$ avec $\mathbf{E}_{t}=\mathbf{\hat{x}\times\left(\mathbf{E}\times\hat{x}\right)}$.}.
% \begin{equation}
% \mathbf{E}_{z}^{i}e^{-j\mathbf{k}^{i}\cdot\mathbf{r}}+\mathbf{E}_{z}^{r}e^{-j\mathbf{k}^{r}\cdot\mathbf{r}}=\mathbf{E}_{z}^{t}e^{-j\mathbf{k}^{t}\cdot\mathbf{r}}
% \end{equation}
% de plus pour que les deux cot�s correspondent en tous points $x$
% et $y$ (\emph{phase matching}):
% \begin{equation}
% e^{-j\mathbf{k}^{i}\cdot\mathbf{r}}=e^{-j\mathbf{k}^{r}\cdot\mathbf{r}}=e^{-j\mathbf{k}^{t}\cdot\mathbf{r}}\quad\mbox{pour }x=0
% \end{equation}
% ce qui n�cessite
% \begin{eqnarray}
% k_{z}^{i} & =k_{z}^{r} & =k_{z}^{t}\\
% k_{y}^{i} & =k_{y}^{r} & =k_{y}^{t}
% \end{eqnarray}
% Puisque par hypoth�se $k_{y}^{i}=0$, alors toutes les composantes
% selon $y$ des vecteurs d'onde sont nulles, impliquant que les plans
% d'incidente et de r�flexion correspondent au plan $xOz$. D'autre
% part, puisque le milieu des ondes incidentes et r�fl�chies est le
% m�me, on a �galement $\theta^{i}=\theta^{r}$. 
% 
% \begin{figure}[h]
% \begin{centering}
% \includegraphics[width=0.6\textwidth]{figures/geometrie_reflexionTransmission_plasma}\caption{G�om�trie du probl�me.\label{fig:G=0000E9om=0000E9trie-reflexion}}
% \par\end{centering}
% \end{figure}
% 
% Au voisinage des fr�quences LH, le milieu plasma peut s'approcher
% par un milieu di�lectrique uniaxe de tenseur de permittivit� relative
% \begin{equation}
% \mathbb{K}=\left(\begin{array}{ccc}
% S & 0 & 0\\
% 0 & S & 0\\
% 0 & 0 & P
% \end{array}\right)=S\left(\mathbf{\hat{x}\mathbf{\hat{x}}+\mathbf{\hat{y}}\mathbf{\hat{y}}}\right)+P\mathbf{\,\hat{z}}\mathbf{\hat{z}}
% \end{equation}
% 
% Dans un milieu bir�fringent, le vecteur de Poynting n'est pas dirig�
% selon le vecteur d'onde $\mathbf{k}$ et le champ �lectrique n'est
% pas orthogonal � $\mathbf{k}$. La simple relation de dispersion $k=n\frac{\omega}{c}$
% n'est donc plus valide. La relation de dispersion d'un tel milieu
% donne, on l'a vu, les deux solutions, donn�es par exemple par l'�quation
% \ref{eq:cold_plasma_dispersion_relation_solution_n}. En ce qui concerne
% l'onde lente, l'indice perpendiculaire selon $x$ a pour expression
% d'apr�s la solution de l'�quation de dispersion :
% \begin{equation}
% n_{x}^{2}=-\frac{P}{S}\left(n_{z}^{2}-S\right)
% \end{equation}
% 
% En utilisant l'�galit� tir�e de la condition de \emph{phase matching}
% on a alors
% \begin{equation}
% n_{x}^{2}=-\frac{P}{S}\left(S+\varepsilon^{i}\cos^{2}\theta^{i}\right)\rightarrow n_{x}=\pm\sqrt{\frac{\left|P\right|}{S}\left(\varepsilon^{i}\cos^{2}\theta^{i}-S\right)}
% \end{equation}
% La direction du nombre d'onde dans le plasma et le signe de l'�quation
% pr�c�dente reste � d�terminer. Pour choisir le signe correct, on doit
% choisir le signe du flux de puissance (vecteur de Poynting) depuis
% l'interface vers le plasma. D'apr�s l'�quation d'onde, on d�duit
% \begin{eqnarray}
% n_{x}n_{z}E_{x}^{t} & = & \left(P+n_{x}^{2}\right)E_{z}^{t}\\
% \end{eqnarray}
% et en utilisant les conditions de continuit� du champ �lectrique et
% de la densit� flux �lectrique: 
% \begin{eqnarray}
% \left(E^{r}-E^{i}\right)\sin\theta^{i} & = & E_{z}^{t}\\
% \varepsilon^{i}\left(E^{r}+E^{i}\right)\cos\theta^{i} & = & SE_{x}^{t}
% \end{eqnarray}
% On dispose maintenant de trois �quations et trois inconnues ($E_{x}^{t},E_{z}^{t},E^{r}$)
% qui nous permet de r�soudre le probl�me: 
% \begin{eqnarray}
% E_{x}^{t} & = & \frac{P+n_{x}^{2}}{n_{x}n_{z}}E_{z}^{t}\\
% E_{z}^{t} & = & \left(E^{r}-E^{i}\right)\sin\theta^{i}\\
% E^{r} & = & ...
% \end{eqnarray}
%  Pour que le flux de puissance soit positif, on doit avoir
% \begin{equation}
% \mathbf{\hat{x}\cdot}\mathbf{S}^{t}>0\rightarrow\mathbf{\hat{x}\cdot\left(\mathbf{E}^{t}\times\mathbf{H}^{t}\right)/2}
% \end{equation}
% or,
% 
% \begin{eqnarray}
% \mathbf{\hat{x}\cdot}\mathbf{S}^{t} & = & \mathbf{\hat{x}\cdot\left(\mathbf{E}^{t}\times\mathbf{H}^{t}\right)/2}\\
%  & = & \mathbf{H}^{t}\cdot\left(\mathbf{\hat{x}\times}\mathbf{E}^{t}\right)/2\\
%  & = & \frac{1}{\omega\mu_{0}}\left(k_{z}E_{x}^{t}-k_{x}E_{z}^{t}\right)\cdot\left(-E_{z}^{t}\right)/2\mathbf{\hat{y}}\\
%  & = & \frac{-k_{0}}{\omega\mu_{0}}\left(\frac{P+n_{x}^{2}}{n_{x}}-n_{x}\right)\cdot\left(E_{z}^{t}\right)^{2}/2\mathbf{\hat{y}}\\
%  & = & \frac{-k_{0}}{\omega\mu_{0}}\frac{P}{n_{x}}\cdot\left(E_{z}^{t}\right)^{2}/2\mathbf{\hat{y}}
% \end{eqnarray}
% 
% Pour que la derni�re expression soit positive, il est n�cessaire que
% $n_{x}>0$ (car $P<0$). C'est donc la racine positive qui doit �tre
% choisie ==> Probl�me: cela devrait �tre la n�gative !:
% \begin{equation}
% n_{x}=\pm\sqrt{\frac{\left|P\right|}{S}\left(\varepsilon^{i}\cos^{2}\theta^{i}-S\right)}
% \end{equation}
% 
% Cela d�montre que l'onde est backward, dans la mesure o� 

\chapter{Lower Hybrid and Current Drive}


\section{Introduction}
Originally, the occurrence of a wave resonance, the lower hybrid resonance, has been anticipated to lead to strong wave-particle interaction through linear and non-linear mode conversion to a hot plasma wave \parencite{Stix1992}. With an appropriate RF launcher conceived to excite cold plasma waves, these would propagate into the plasma until reaching the lower hybrid resonant layer at $\omega_{LH}$. This resonance exists in tokamak plasma in the region close to the ion plasma frequency $\omega \pi/2\pi$ , which lies in the lower end of the microwave band (1-5 GHz). At this layer, the perpendicular group velocity vanishes and the waves can convert into a hot plasma mode which is absorbed. This heating technique, known as Lower Hybrid plasma Ion Heating (LHIH) or Lower Hybrid Resonance Heating (LHRH), was experimentally investigated in the 70' \parencite{Bellan1974, Hooke1972, Golant1972, TONON1977, Schuss1981}. Different physical mechanisms have been invoked to explain the energy absorption, such as stochastic Ion Heating\parencite{Karney1977b,,Karney1978a, Karney1979c} and quasi-linear electron Landau damping \parencite{Brambilla1983, Fisch1978, Karney1979}.

In the 80', effective ion heating had only been obtained in a small number of experiments and research along the application of LH waves towards bulk ion heating were slowing down \parencite{Gormezano1986, Porkolab1984a, Tonon1984}. The reason for this is that bulk ion heating near the mode conversion layer appeared to be less reproducible and more difficult to achieve than electron heating. Indeed, as the wave frequency gets closer to the lower hybrid frequency, the shorter wavelength waves may be more effectively absorbed and/or scattered near the plasma surface by nonlinear effects such as parametric instabilities, low-frequency fluctuations, etc. Moreover, for LH bulk ion heating, the unconfined ions impinging on the wall induced a large amount of metallic impurities and then the increase of power radiated by the plasma.

Rather than trying to heat ions, it was theorized postulated that high phase velocity waves traveling in the direction parallel to the magnetic field could interact quasi-linearly by Landau interaction with the electrons population, and, by using an asymmetric spectrum could drive a large amount additional of toroidal plasma current \parencite{Fisch1978}. In the same fashion that for LHRH, the RF power is coupled to the plasma via launchers made of rectangular waveguides stacked periodically in the horizontal direction parallel to the toroïdal magnetic field. However, at the contrary of LHRH launchers, the LH waves are launched preferentially in one toroidal direction by mean of a phased array. The LH wave excited by such an array has an asymmetric parallel spectrum. The LH waves create an asymmetry in the electron distribution, which ultimately results in a net electric current \parencite{Fisch1987}. This technique is known as Lower Hybrid Current Drive (LHCD) and has been confirmed on the PLT tokamak in 1982 \parencite{Bernabei1982, Motley1985, Jobes1985} and in Alcator C in 1984 \parencite{Porkolab1984}. 

Since in 1982 many impressive results were presented on LHCD\parencite{Hooke1982, Porkolab1982, Tonon1982} toward steady state or quasi steady state tokamak operations, most LH experiments were dedicated to electron interaction and especially to current drive. 

Despite the fact that the Lower Hybrid resonance is not anymore involved to use of current LH launchers in tokamaks, the term remained. Currently, the Lower Hybrid waves term refers to the waves which satisfy the slow-wave branch of the cold plasma dispersion relation for parallel index larger than one ($|n_{\parallel}|>1$) and a RF frequency $\omega$ which lies between the ion cyclotron $\omega_{ci}$ and the electron cyclotron $\omega_{ce}$ frequencies. 

For the LH (lower hybrid) method which operates at the lower end of the microwave band (1-5 GHz) klystrons transform electrical power into electromagnetic power (step 1), which is transported to the plasma using waveguides (step 2). The power is coupled to the plasma with antennas, which because of their characteristic shape are called grills (step 3), transported inside the plasma by plasma waves (typically the slow wave) (step 4), and absorbed on ions or electrons by wave-particle interaction (step 5).

\section{Klystrons}
A klystron is a vacuum tube used as amplifiers at narrow band microwave and radio frequencies. In the fusion domain, they are used mainly to produce high power waves, at the level of hundred of kilo watts during many seconds. First klystrons have been invented by the Varian brothers in 1937 \parencite{Pond2008}. The first klystrons have been intensively used during World War II, as RF power generators for RADAR systems. Klystrons sources between 2.45 GHz and 5 GHz are now available at the 0.5-0.8 MW/10-1000 s level\parencite{Pond2008}. 

The basic idea of the klystron is illustrated in Figure 1 \parencite{Tenenbaum2003}.

\begin{enumerate}
 \item  A continuous electron beam is emitted by the klystron's cathode and accelerated to high voltage in a DC gun 
 \item The electron beam passes through a resonant cavity (input cavity) which is excited by an external source of RF power (typically in the miniwatt to watt range) at the resonant frequency of the cavity (resonant cavity) 
\item While passing through the first cavity, the electron beam is velocity is modulated by the weak RF signal: the sinusoidal variation in the cavity voltage causes alternate acceleration and deceleration of electrons in the beam. 
\item The beam passes through a drift tube, in which the accelerated electrons travel faster and the decelerated ones travel slower, resulting in a beam which is bunched at the frequency of the RF drive signal; thus a significant fraction of the beam’s power has been moved from DC to the drive frequency. 
\item The beam passes through a series of cavities in which are induced standing waves at the same frequency as the input signal. The signal induced in the second chamber is much stronger than that in the first, thus leading to amplification of the velocity and density modulation of the beam. A larger number of cavities may be used to increase the gain of the klystron, or to increase the bandwidth. 
\item The output cavity is located at a point where the electron bunches are fully formed. The resonant frequency of the output cavity is the same that the input cavity and the power is transferred from the beam to the output cavity. 
\item The RF power extracted from the output cavitiesy travels to an output waveguide(s) and exits the klystron body.;  since the klystron us vacuum pumped, a RF window is necessary. RF feedtrought (or “windows”) are mounted at the RF outputs to ensure the transition between the vacuum inside the tube and the pressurized waveguides.
\item the The residual electron beam energy is dissipated in an activetly cooled collector. spent beam is transported to a collector, which is a water-cooled beam stop. 
\end{enumerate}

Figure 1 Schematic cut of a klystron


Figure 2 Two Klystrons used in the Tore Supra tokamak LH system. Left: 500kW/210s klystron (1980'); right: 700kW/CW klystron (2002). Both have been manufactured by Thales Electron Device. 16 of each model have been installed in the Tore Supra LH plant. These klytrons require a high voltage supply of 60 to 76kV (cathode voltage) for 2220-22 A (beam current). Their gain is greater than 50 dB for an efficiency of 40-45\%.

\section{LH Transmission Lines}
As the RF frequency increases, coaxial cable losses (expressed in decibel per distance in Figure 3) become unpracticable for high power applications. In the LH range of frequencies (1-5 GHz), hollow rectangular waveguides are preferred for the transport of the RF power from the klystrons to the torus. In addition to their low losses, they also allow a greater breakdown voltage than coaxial lines of the same size. In this frequency range the wavelength in vacuum is of the order of 30 cm to 6 cm.

Figure 3. Various transmission line attenuation vs frequency. Source: http://www.rfcafe.com/references/electrical/ew-radar-handbook/microwave-waveguide-coaxial-cable.htm

\subsection{Rectangular waveguides} 
A hollow rectangular waveguide of width a and height b is illustrated in Figure 4. Solving Maxwell’s equations for this geometry leads to multiple possible solutions, or modes. These modes are eingenfunctions of the equation system. Each mode is characterized by a cutoff frequency, below which the mode cannot propagate in the guide. In a rectangular waveguide, modes can be expressed as Tranverse Electric (TE) or Transverse Magnetic (TM), depending of their respective polarization. Since the walls of a rectangular waveguide constrain the electromagnetic field boundary conditions along two dimensions, two integer indices are used to describe a mode from another. Thus, modes in a rectangular waveguides can be TE10, TE20, TM11, etc. 

For practical applications, the dimensions of the waveguides are generally chosen in order to have one and only one mode allowed propagating for a specified frequency band. This single mode is the first one to appear to be propagating, and is referred as the fundamental mode (generally labelled TE10). Other modes can eventually be excited by waveguides discontinuities, but can’t propagate since they are evanescent. They are referred to high order modes.
For high power applications, a great care is given to waveguide inner walls, bends and connections, since reflected power and arcs may occur due to discontinuities in the conducting walls, such as the ones caused by misalignments, bumps, holes, etc.

Figure 4. Illustration of a rectangular hollow waveguide. Source:Wikipedia.

\subsection{Waveguide plumbery – a practical example with Tore Supra}
Many waveguide devices have been developed in order to transmit, measure, combine or split the power from one point to another. These devices are commonly used on LHCD systems to transport the power from a klystron to (a section of) an antenna, but also in order to protect the klystron from possible reflected RF power by the plasma.   

As an example of the usage of such devices, the Tore Supra LHCD system is illustrated in Figure 5. The power is generated at the klystron plant and transmitted to the two launchers through rectangular waveguides (8 lines per launcher). 


Figure 5.Schematic of the Tore Supra LHCD system, from the klystrons plant to the vacuum vessel. 

Once reaching the launcher rear end, the power goes trough a hybrid junction (also called 3-dB splitter). This device splits the incident power into two waveguides, one to feed the upper part and the other for the lower part of the launcher.  If reflected power (from the plasma) returns from these two waveguides, the power is recombined and directed to a fourth one. An actively cooled water load is connected to this fourth port, in order to dump the remaining RF power reflected by the plasma and thus protect the klystron. A rectangular waveguide 3-dB splitter is illustrated in Figure 6. 

Figure 6. Top: Illustration of an rectangular waveguide hybrid junction. When excited from port 1, the power is split to ports 2 and 3. The electric field distribution is also illustrated.Bottom: when the power comes from ports 2 and 3, the power is recombined and directed to port 4, where the power is dumped into a water load. 

After being splitted, the RF power goes through various devices up to the plasma (Figure 7): 
\begin{itemize}
\item A bidirectional coupler, which allows measuring the forward and reflected power amplitude and phase ;
\item A DC break, which isolates the DC potential of the tokamak from the DC potential of the transmission line (for the protection of persons);
\item A vacuum feed through (window) which isolates the tokamak vacuum from the pressurized medium existing in the waveguide. These windows (16 by launchers) are illustrated in Figure 7. A window consists in a dielectric medium inserted between two rectangular waveguides. The dimensions of the dielectric medium (a ceramic, such beryllium oxide BeO or alumina) are calculated in order not producing reflected power. A part of the power is always absorbed in the ceramic, so a great care is applied to correctly cool the window in order to avoid heating up and ultimately break the ceramic.
\item A TE10-TE30 mode converter, which splits the incoming RF power into three, thus leading to split the power into three poloidal rows of waveguides. A mode converter is a rectangular waveguide device which large side is modulated in order to allow higher modes (such as TE20, TE30 and TE40) and such that the power is totally transferred from one mode to another one (efficiency is close to 100\%). Once the power is transferred to the TE30 mode, metallic septa located in zero field regions split the power into three independent waveguides. The device is illustrated in Figure 8.
\item Finally a multijunction, a structure which will be described in details in the next section, splits and phases the power in the toroidal direction, then leads to the plasma. 
\end{itemize}



Figure 7.Illustration of the power splitting scheme in Tore Supra LHCD launcher C3 or C4.  


Figure 8. Illustration of the electric field distribution in a TE10-TE30 mode converter at 5 GHz. The device is excited from the left.

\section{Grill LHCD launchers}
\subsection{“Grill” launchers}
To couple the power from the waveguides to the plasma an open ended waveguides are used. Since As the plasma cross-section dimensions of toroidal devices of the time are became sufficiently large to accommodate the free space wavelength at the lower hybrid frequencies, it becomes possible to couple the RF power directly by open-ended waveguides inserted in the vacuum vessel wall, thus avoiding coils or antennas built within the vacuum chamber\footnote{To give some numerical orders, the COMPASS-D tokamak, operated in Culham from to 1989 to 2001, used 1.3 GHz RF power sources, and rectangular waveguides of cross-section 165x82mm to carry the RF power to the LH launcher. Smaller wavelengths (i.e. higher frequency sources, up to 8 GHz) have been used, leading to smaller waveguides cross-section.}.
 
However, in order for the wave to access from outside the plasma to inside the plasma, the wave has to be “slowed down” along the static toroidal magnetic field. This means that the parallel phase velocity of the wave $v_{\parallel}$ has to be reduced with respect to the speed of light $c_0$, in order to being able to penetrate trough the plasma. It is said that the waves must satisfy an accessibility condition \parencite{Golant1972, Stix1992}\footnote{The accessibility condition is sometime referred as the Golant's accessibility criterion, for e.g. in\parencite{Puri1974}.}

Equivalently, this condition requires for the wave to have a sufficiently large refractive index $n_{\parallel} = c_0/v_{\parallel} = k_{\parallel}/k_0$ parallel to the magnetic field. The use of a phased array as a way to slow down the waves and thus to launch LH waves was first suggested by Lallia\parencite{Lallia1975} and investigated by Bernabei et al. in 1975\parencite{Bernabei1975}. Lallia suggested that a set of properly phased waveguides, called “grill”\footnote{The similarity to the cooking utensil leads to nickname these launchers “grill”.}, would efficiently slow down the wave parallel to toroidal magnetic field. The short sides of the waveguides are mounted parallel to the toroidal magnetic field, to launch effectively a slow wave mode into the plasma. 

The coupling is the process by which RF power propagates and possibly tunnels from an antenna to the plasma \parencite{Meneghini2012}. Brambilla developed the theory of the coupling of  a simplified grill structure to a large toroidal plasma \parencite{Brambilla1979} \footnote{The Brambilla theory has been applied (and still) to many different geometries, from single waveguide to hundred of waveguides array. Some examples are cited here, like a 2-waveguides array with one mode in \parencite{Krapchev1978} or to 16-waveguides array in \parencite{Stevens1988}.}

This coupling theory has been generalized later to the fast wave coupling \parencite{Theilhaber1979,Theilhaber1980} and to 3D rectangular waveguides in \parencite{Bers1981, Bers1983}. 

Several important remarks can be drawn from the properties of the propagation of cold-plasma waves into the plasma. The wave is evanescent if the electron density is below a cut-off density $n_{\mathrm{cutoff}}$. However, if such a low density layer is thin enough, the wave can tunnel through it. Thus, in order to efficiently couple high power to the plasma, the LH launcher must be close enough of the plasma, similarly to ICRF antennas. 

In these conditions, the thermal and mechanical stresses are essential parameters to design an LH launcher. They and constitute a challenge to deliver such for these systems in for fusion reactors. As for ICRH antennas, LHCD launchers distributed in the vacuum vessel is being proposed for DEMO, mitigating substantially these issues.

Different types of grills have been developed to accommodate the conditions that the wave must be coupled to and (then) propagate in the plasma. 


Figure 9. Picture of the grill mouth of the COMPASS-D LH launcher, made of 8 waveguides. Generator frequency was 1.3 GHz.


Figure 10. The Alcator C-Mod LH1 launcher (88 waveguides, 4.6 GHz) and it associated transmission line (aka the”jungle gym”). 

\subsection{Multijunction Launchers}
In a large tokamak, a conventional LH grill made of independently fed waveguides would require hundred of waveguides, leading to a very complex power splitting design behind the antenna (cf.Figure 10). The use of \emph{multijunction} grill was suggested in 1984 to overcome this limitation \parencite{Gormezano1985, Moreau1984}. In a multijunction grill, the main waveguide is divided into $N$ smaller (secondary waveguides) by thin metallic walls parallels to the wall of the main waveguide and perpendicular to the electric field: the \emph{E-plane N-junctions}. Built-in phase shifters made by reducing the waveguide height (in order to increase the guided wave phase velocity) are added in the structure in order to obtain the desired output phasing of the grill. Multijunction launcher makes it simpler to create and feed a large number or waveguides, at the contrary of classic grills launchers. However, a drawback is that the adjustment range of the power density spectrum is limited to a smaller range than for classic grills launchers.
A judicious choice of the phasing of the output waveguides leads to an important self-matching property of the multijunction (also known asor recycling effect or load resilience). Indeed, for specific phase values, the reflected waves from the plasma which return back in the secondary waveguides can be reflected back to the plasma, thus leading to multiple reflections in the secondary waveguides. This recycling effect, which takes place between the plasma-antenna discontinuity and the E-plane bi-junctions, leads to an attenuation of the waves at each passage by the plasma, and ultimately to a decrease of the reflected power toward the RF sources. 
TODO: Illustration of the recycling effect (load resilience) inside a multijunction.


Figure 11 Left:The Tore Supra Multijunction launchers as view from inside the vacuum vessel. The C2 launcher (128 waveguides, total dimensions 50 x 40 cm, 1989). Right: The C3 launcher (288 waveguides, total dimensions 60cm x 60cm, 1999). Frequency: 3.7GHz. 

\subsection{Passive-Active waveguide array launchers}
One of the main showstopper to the use of LHCD to realize long and performant pulses (outside the constraints of the tokamak magnetic coils heating) is the ability to couple high LH power to the plasma. Indeed, in order to achieve good coupling, the electron density in front of the LH launchers needs to be large enough, which means that the launcher needs to be located close enough to the plasma or to increase the local density by gas puffing means. 
The alternation of passive waveguide (a waveguide closed by a short-circuited) and active waveguide (a waveguide that is directly feed from the generator) has been initially proposed by Motley and Hooke \parencite{Motley1980} in order to minimize the surface wave excitation. Moreover, it was found that adding passive waveguide at each side of the array leads to decrease the reflected power in the last active guide \parencite{Krapchev1978, Motley1980B}. The passive waveguides act as reflectors, radiating back a part of the power reflected by the plasma, and thus improving the coupling efficiency. Passive-Active grills were envisaged since the 1980' for fusion-reactor grade devices \parencite{Ehst1982}. In these first conceptual designs, the passive waveguides were thought to be plugged or tuned individually in order for the outgoing power to have the requisite phase corresponding to an all-active grill system. 

Being close to the plasma leads to increase the heat fluxes into on the launcher front face. Thus, in order to improve the cooling of the launcher front face, it has been proposed in 1995 to insert one passive waveguide between each active waveguide, behind which a water pipe could efficiently water-cool the structure \parencite{Bibet1995}. 

In order to insure a lower reflected power than a conventional grill, it has been proposed to associate this alternation of passive-active waveguide to  use a multijunction. This Passive-Active Multijunction (PAM) concept adresses two of the main criticism made to LH launchers, i.e. the coupling efficiency and the heat loads resilience. 

Figure 12 PAM prototype tested on the FTU tokamak (24 active waveguides, 24 passive waveguides, 8 GHz, 2003) \parencite{Mirizzi2003, Ridolfini2005}.


Figure 13 The Tore Supra PAM (C4) launcher (96 active waveguides, 102 passive waveguides, approx. 7 tons, front face dimensions 60cm x 60cm, 3.7 GHz, 2009) \parencite{Guilhem2009, Guilhem2011}.

\subsection{Poloidal splitter launchers}
Motivated by the reduction of the RF losses and the increase of the reliability of a LH launcher, while keeping the flexibility in the parallel index $n_{\parallel}$ spectrum of the “grill” configuration (at the contrary of multijunction), the Alcator C-Mod team developed in 2008 a launcher (LH2) based on a four way splitter \parencite{Koert2008a}. At the contrary of multijunction launcher, the four way splitter divides the power in the poloidal direction. Since the power splitting is done poloidally, the launched parallel power spectrum can be changed with the same flexibility of conventional grill antennas. This design simplified the feeding structure of the antenna, thus reducing the RF losses due to multiple power splitters and flanges of previous grill configurations, while keeping a simple manufacturing assembly. The RF power is redistributed depending on the plasma load on each of the four rows of the splitter. If the load is the same for each row, the power is evenly split among rows. The RF optimization of the launcher has been made for a source frequency of 4.6 GHz and assuming a simplified plasma load (constant) in front of each row. A similar design has been made for the KSTAR tokamak, at 5 GHz \parencite{Kim2012}. 

Figure 14:Left: CAD conceptual schematics of the four-way splitter of Alcator C-Mod \parencite{Koert2008a}.Right: final assembly CAD model with 16 four-way splitters \parencite{Meneghini2010b}. 

Figure 15 Picture of the LH2 launcher inside the Alcator C-Mod vacuum vessel with its lateral protections in January 2012 \parencite{Meneghini2010b}.

\section{Wave propagation and absorption in the plasma}
In order for the RF power to reach the plasma core, the excited power density spectrum excited by a the gril LHCD launcher antenna must satisfy different criterions: 

\begin{itemize}
\item  A cut-off condition: the electron density in front of the launcher must be close or higher than the slow-wave cut-off density.
\item A propagation condition: in order for the LH waves to propagate into the magnetized plasma, excited waves must have an absolute value of the parallel index greater than one, i.e. $|n_{\parallel}|>1$ 
\item An accessibility condition: in order for the slow waves which access to the core plasma not to be mode-converted into fast waves, a minimum value of $n_{\parallel}$ for a given plasma density and magnetic field is requested :
$|n_{\parallel} |>| n_{\parallel, \mathrm{access}} | \approx \sqrt{1- \omega \pi2 /\omega^2 + \omega_{pe}^2 /\omega_{ce}^2 } + \frac{\omega_{pe} }{| \omega_{ce} |}$ 
When the accessibility condition is satisfied, the slow and the fast branch of the dispersion relation are separated and the LH waves can reach the core plasma.
Inversely, for a given $n_{\parallel}$, the previous condition leads to an upper limit for the density, above which the wave can’t propagate.
\end{itemize}

Figure 16.Illustration of the power density radiated by a LH launcher (seen into a horizontal plane cut) into a plasma of increasing electron density. The plasma current direction is indicated. Since the launcher is designed to launch a non-symmetrical power spectrum, the power flows toward a preferred direction (opposite to the plasma current). 
The RF power is coupled by the launcher to the plasma, which means that the electromagnetic waves in the launcher’s waveguides are converted to (mainly) slow waves in the plasma. In the LH range of frequencies (~2 to 8 GHz), the wavelength of the RF waves in the plasma is well below the typical beam size, which itself is inferior to the equilibrium non-uniformity scale. It this situation, the evolution of the waves in the plasma can be described by the ray-tracing formalism. Since the evolution of the electron population can be described with Fokker-Planck calculation, the combination of the both tools is a standard for modeling LHCD experiments.

As said before, the LHCD launcher is a phased array of waveguide which excites a slow plasma wave. The following calculations taken from current drive theory illustrate the basic requirements of the LHCD launcher in order to optimize the current drive efficiency. 

In the plasma, the incremental current $\Delta j$ carried by an electron of electrical charge $q=-e$ and accelerated from a parallel velocity $v_\parallel$ to $v_\parallel + \Delta v_\parallel$ is:
$$\Delta j = q \Delta v_{\parallel}$$

The incremental increase of kinetic energy is:
$$\Delta E = m v_{\parallel} \Delta v_{\parallel}$$

Eliminating $\Delta v_{\parallel}$ leads to an incremental current: 
$$\Delta j = \Delta E q/(m v_{\parallel})$$ 

Let $\nu_\mathrm{coll}$ be the Coulomb collisional frequency. At first order5, the incremental energy input $\Delta E$ persists for a time $1/\nu_\mathrm{coll}$. The associated (steady-state) RF input power $\Delta P$ required is thus: 
$$\Delta P = \nu_\mathrm{coll} \Delta E$$ 

From the two previous equations, the ratio of the incremental current to the RF input power is:
$$\Delta j/ \Delta P = q / (m \nu_\mathrm{coll} v_{\parallel})$$

This suggests that more current can be driven with low parallel velocity electrons. However, fast electrons collide less often than slower, as the Coulomb collision cross section falls off with increasing relative velocity as $\nu_\mathrm{coll} \propto n_e/v_{\parallel}$3 (for parallel velocity larger than the thermal velocity). It follows that the previous ratio is proportional to: 

$$\Delta j/ \Delta P \propto (v_{\parallel}^2 q) / (m n_e)$$

this means that high current drive efficiency can be reached from fast electrons. Thus, it is actually more effective to push fast electrons than slower. Even if it may be energetically more expensive to accelerate fast electrons, this energy deposition need occurs less often because current last longer when carried by relatively less collisional electrons \parencite{Fisch1987}. 

For the purpose of current drive with LH waves, the Landau damping is the dominant absorption mechanism. Landau damping is a collisionless damping process in which particles exchange energy with waves traveling with the nearly same phase velocity parallel to the magnetic field, that is, for particle parallel velocity $v_{\parallel}$ satisfying resonant condition:

$$\omega – k_{\parallel} v_{\parallel} = 0 $$ 

or 

where $k_{\parallel}=k_0 n_{\parallel}$ is the parallel wavenumber of the wave, $n_{\parallel}$ the parallel index of refraction and $k_0= \omega/c$ the wavenumber in vacuum. From the previous relation one deduces that the LHCD launcher must excite waves satisfying the resonant condition $v_{\parallel}=c/n_{\parallel}$. As for the slow wave to be able to penetrate into the plasma the parallel index must be greater than one and also greater to $|n_{\parallel, \mathrm{access}}|$, typical LHCD launchers in current tokamaks excite a main parallel index between $n_{\parallel 0}$=1.5 to 3.0. 

However, in many past and present LHCD experiments, the resonant velocity $v_{\parallel 0}=c/n_{\parallel 0}$ corresponds to suprathermal region where the number of electrons should be in principle too small for any significant wave damping to take place and to account for the observed current drive. Indeed, strong wave damping on a Maxwellian distribution with temperature $T$ requires the wave damping to be no larger than four times the thermal velocity $v_T=\sqrt{k T/m}$. This paradox is commonly referred to as the spectral gap problem and is an active area of research. Various explanations are proposed to explain this spectral gap, among these the toroidal effects on the wave propagation that can cause sufficient upshift (increase of the parallel wavenumber) in the parallel refraction index $n_{\parallel}$ to “fill” the spectral gap, nonlinear interactions such as parametric decay instabilities (PDI), diffraction effects or power density spectrum fluctuations. 


\section{Overview of parameters achieved and example of some systems}
\subsection{LH Launchers main figures of merit}

There are three important figures of merit to measure the efficiency and the performances of a LH launcher: 
1. The first one is the k-space radiated power spectrum, generally characterized by its parallel power spectrum $p(n_{\parallel})$ . This quantity represents the amount of power for each parallel index excited by the launcher. The relation between this power spectrum and the array excitation is related to the Fourier transform of the electromagnetic field at the plasma-antenna interface. 
2. The second one is the ratio of the reflected power (at the mouth or at the end of the launcher) to the input power, aka the reflection coefficient (sometime expressed in percent). 
$RC = P_r/P_i$
3. The third one is the directivity of the launcher, which is the fraction of the power spectrum over its total power content. It can be viewed as the fraction of the power that goes toward one toroidal direction over the total coupled power. One can define the directivity as follow6
$$D= \frac{\int n_{\parallel} >0 p(n_{\parallel}) dn_{\parallel} }{\int n_{\parallel} p(n_{\parallel}) dn_{\parallel} } $$

Theoretically, if the $n_{\parallel}$ spectrum of an antenna is known, one can evaluate the reflection coefficient. However, since the wave field at the launcher aperture (and thus the launcher spectrum) depends on the plasma itself, numerical codes are required to make a self-consistent numerical evaluation of the coupling. Such codes are coupling codes.

\subsection{LH systems and launchers performances}
Mainly because of its relative simplicity, most of the LH launchers have been based on a “grill” configuration, as for example in the 1980' \parencite{Porkolab1984a, Gormezano1986a, Stevens1988}\footnote{It is a non exhaustive list.}

Typical grill launchers have a high reflection coefficient (of the order of 20 to 40\%) but directivity higher than 80\%. Since all the waveguides can be feed by independent RF sources, and thus independently phased, such launchers have a wide flexibility in terms of operational space (excited spectrum) which is of great interest for physics studies. However, since each waveguide is independently power fed, the complexity of the launcher grows enormously with the number of output waveguides, leading to cumbersome transmission systems for multi-megawatt power levels. 

Table 2.1 Waveguide number and maximum performances in some grill launchers.
Launcher
Total number of waveguides
Max power [MW]
PLT Grill 1 (1981-1984)  \parencite{Stevens1988}
6 (0.8 GHz)
TODO
PLT Grill 4 (1986) \parencite{Stevens1988}
16 (2.45 GHz)
TODO
COMPASS-D (1989)
8 (1.3 GHz)
TODO
Alcator C \parencite{Porkolab1984a}
16 (4.6 GHz)
TODO
Alcator C-Mod LH1 (2006)
88 (4.6 GHz)
TODO

Multijunction launchers at the contrary have led to hundred of waveguides launchers (see Table 2). The complexity of the multijunction design and manufacturing is balanced by the a simpler transmission line system behind the launcher. Because of the self-matching property, the typical reflection coefficient of multijunction launchers is generally less than 10\%. However, the multiple back and forth of the waves leads to an increase of the peak electric field in secondary waveguides. Moreover, since the phase shift is created by the built-in phase shifter inside the launcher, the flexibility of phase configuration is reduced in comparison to classic grill launchers. 

Table 2.2 Waveguide number and maximum performances in some multijunction launchers 
Launcher
Total number of waveguides
Max power [MW]
Petula-B \parencite{Gormezano1985}
4 (1.3 GHz)
TODO
JT60 CD-3 \parencite{Ikeda1989}
24x4=96 (1.74-2.23 GHz)
TODO
JET \parencite{Litaudon1990a}
2x2x2x6x8=384 (3.7 GHz)
TODO
Tore Supra C1 and C2 \parencite{Litaudon1992a}
4x2x8x2=128 (3.7 GHz)
TODO
Tore Supra C3 \parencite{Bibet2000}
6x3x8x2=288 (3.7 GHz)
TODO
EAST
8x4x5=160 (2.45 GHz)
TODO

The PAM concept tested in the Tore Supra tokamak showed that reflected power lower than 5\% (i.e. high coupling efficiency) and continuous operations could be combined (cf. Figure 17) during long pulse operations. 

Figure 17 Illustration of a Tore Supra discharge with the PAM launcher (TS\#45472, 2010) \parencite{Ekedahl2010b}. The design goal power density (25MW/m$^2$ ) is achieved during 78s, with very low reflected power (<2\% of the input power), even at large plasma launcher distance (~10 cm) (density is not the line-average density). The Infra-Red monitoring shows that the global temperature of the launcher mouth is lower than 270$^{\circ}$C, validating its efficient cooling structure.

\section{ITER System}

Although not part of the initial ITER planned procurement phase, a 20 MW/CW 5 GHz LHCD system is due to be commissioned and used for the second mission of ITER, i.e. Q=5 steady state target \parencite{Hoang2009}. In ITER, the parallel index of the launched waves $n_{\parallel 0}$ ~ 2.0 has been selected as a trade-off between the current drive efficiency, the wave accessibility and the location of power deposition, all of which depend upon the plasma conditions. Depending of the plasma scenarios (steady-state, hybrid, ramp-up, etc.), the additional current driven by LHCD would range between 0.42 MA (steady-state) up to more than 3.0 MA (ramp-up) \parencite{Decker2011}. In any case, these values confirm that for steady state scenarios, a substantial bootstrap current is required \parencite{ITERPhysics_Chap6}. The average current drive efficiency, which is the figure of merit of the LHCD, has been calculated to be $\eta \equiv 0.2 \times 10^{20} A m^{-2} W^{-1}$, similar to the ones measured in present days tokamaks \parencite{ITERPhysics_Chap6}. This value can be compared to other additional current drive scheme, such as NBI or ECCD. 

In addition, LHCD-assisted start-up could reduce the flux consumption during current ramp-up, resulting in a longer flat top or burn time. An early application of 20 MW LHCD during the plasma current ramp-up phase of the ITER reference scenario 2 is effective to reduce the flux consumption. An expected saved flux of 43Wb is equivalent to about 500 s of additional burn duration \parencite{Hoang2009}.

\section{Overall discussion on strength and weaknesses}
Nowadays, the main usage of Lower Hybrid waves concerns current drive applications, in relation to quasi or fully steady-state plasma physics and scenario developement. For this reason, LHCD systems are commons on supraconducting tokamaks and have a lead on continuous wave operations, from continuous wave RF sources, actively cooled transmission lines and launchers.  

In term of performances achieved, fully non inductive discharges of up to 3.6 MA in JT-60U and 3 MA in JET have been achieved with LHCD. Plasma currents of 0.8 MA and 0.27 MA have been sustained for more than 6 minutes in Tore Supra and EAST. At very low density, the tokamak TRIAM-1M sustains even a 5 hours discharge. The EAST team also demonstrated 30s long pulse H-mode operation and ELMs mitigation possibility using LHCD. The current drive efficiency of LH waves has been measured to be equal or greater of than for other systems.

The coupling of LH waves to the plasma is achevied by launchers, a phased array of rectangular waveguides. The coupling of LH launcher is well understood and code predictions are in good agreement with experiments. As mentioned above, the slow waves excited by the launcher can propagate only if the electron density facing the launcher is greater than a cutoff density, ne cutoff = 3.1 1017 m-3 at 5 GHz for ITER. Launchers are affected by a rapid decrease of the coupling efficiency for lower density, since in this case the wave is evanescent. Multijunction launcher and latter PAM structures solved part of this problem, demonstrating a total reflected power less than few percent in a wide range of edge density. Moreover, additional gas injection in the vicinity of the launcher has also been demonstrated with success. In any case, the launcher still has to be close (~cm) of the plasma edge and thus has to handle large plasma heat fluxes. These heat fluxes can be sustained with actively cooled structure, such as demonstrated in a PAM. 

In a fusion reactor, additional neutron heating fluxes, possible interactions with alpha particles and material requirements (such as avoiding copper) are additional challenges. Distributed launcher designs in the vacuum vessel have been proposed to reduce the power density requirement imposed by the use of fixed-dimensions ports, thus reducing the risk of arcing. From the launcher technological point view, CW operations have been demonstrated but additional margins, required for an efficient use into a reactor, must be gained in terms of reliability. 





\printbibliography[heading=bibintoc]

\end{document}
